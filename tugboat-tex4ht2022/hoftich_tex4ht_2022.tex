% This template file is public domain.
%
% TUGboat class documentation is at:
%   http://mirrors.ctan.org/macros/latex/contrib/tugboat/ltubguid.pdf
% or
%   texdoc tugboat

\documentclass{ltugboat}

\usepackage{microtype}
\usepackage{graphicx}
\usepackage[hidelinks,pdfa]{hyperref}

%%% Start of metadata %%%

\title{What's new in \TeX4ht}

% repeat info for each author; comment out items that don't apply.
\author{Michal Hoftich}
\EDITORnoaddress
% Bellušova 1829, Prague 5, 155 00
\netaddress{michal.h21@gmail.com}
\personalURL{https://www.kodymirus.cz/}
%\ORCID{0}
% To receive a physical copy of the TUGboat issue, please include the
% mailing address we should use, as a comment if you prefer it not be printed.


%%% End of metadata %%%

\newcommand\makefourht{\texttt{make4ht}}

\begin{document}
\maketitle

\begin{abstract}
This article provides an overview of the recent development of \TeX4ht, 
\LaTeX{} to XML converter, and \makefourht, the build system that carry this conversion.
\end{abstract}

\section{Introduction}

Richard Koch wrote an article on interactive documents produced using
\TeX4ht in this issue of \TUB. He and Karl Berry asked if I wanted 
to provide additional tips on the usage of \TeX4ht and also to summarize 
recent changes in the system.


You can find the basic summary of the basic features of \TeX4ht in my previous article
\cite{Hoftich:2019:TLW}. I will focus on the new features and changes in this section.

\section{Changes and new features in \makefourht}

There are some substantial changes in \makefourht\ build system. These are the
most important:

\subsection{Terminal output}

Originally, \makefourht\ showed the full terminal output of \TeX and all 
the commands it called during the conversion process. It resulted in a huge 
amount of information printed on the terminal. It also used the default 
behavior of \LaTeX, so the compilation was stopped for every error, waiting
for the user action.

The new default behavior is to run the compilation in the non-stop mode, 
with most of the terminal output suppressed. Only errors and warnings are 
shown. 

You can change the output method using a new command line option
\verb|--loglevel |, or \verb|--a | in the short form. Each log level prints 
messages of the current level and all higher levels.
It supports the following levels:

\begin{description}
  \item[error] -- print only error messages
  \item[warning] -- show \makefourht\ warnings, for example, from HTML postprocessing filters
  \item[status] -- this is the default level 
  \item[info]  -- print all \makefourht\ messages, but suppress the output from commands
  \item[debug] -- this level is the original default, it prints output of
    \TeX\ and all other executed programs, and it also stops on compilation
    errors.
\end{description}

\subsection{Input redirection}

\makefourht\ now supports shell input redirection, which means that it can 
process the output of other commands without the need to use temporary files.
You need to pass \verb | -| as the filename, and also set the output filename using
the \verb|--jobname| or \verb|-j| option:

\begin{verbatim}
$ python generatetex.py | make4ht -j foo -
\end{verbatim}



\subsection{Conversion of additional markup languages}

In addition to \LaTeX\ and Plain\TeX, \makefourht\ supports some additional
markup languages, thanks to the \verb|preprocess_input| extension. 
It detects the markup used using the file extension, so it is necessary
to name the file accordingly. It preprocesses the input using Pandoc or R with
the Knitr library, which needs to be installed on your system.

The list of supported file extensions is as follows:

\begin{description}
  \item[rtex] -- \LaTeX\ with R code chunks
  \item[rnw] -- \LaTeX\ with the Sweave code chunks
  \item[rmd] -- RMarkdown
  \item[rrst] -- R + reStructuredText
  \item[md] -- Markdown
  \item[rst] -- reStructuredText
\end{description}

For example, the following sample contains \textt{RTeX} document:

\begin{verbatim}
\documentclass{article}
\begin{document}

You can type the R commands into your 
\LaTeX{} document, which will be 
processed and their output included 
in the document:
<<>>=
# Create a sequence of numbers
X = 2:10
# Summary of basic statistical measures
summary(X)
@
\end{document}
\end{verbatim}

You can compile it with the following command, which loads
the \verb|preprocess_input| extension:

\begin{verbatim}
$ make4ht -f html5+preprocess_input x.rtex
\end{verbatim}

\subsection{New commands available in build files}

You can use Lua build files to call additional commands, such as indexing and
bibliography processors, with \makefourht. There are now built-in
commands for Biber, \BibTeX, Makeindex, Xindy, Xindex and PythonTeX. They 
take care of the special settings necessary to work correctly with \TeX4ht.

As an example, the following sample file should produce an index with links 
that point to the place where \verb|\index| were used:

\begin{verbatim}
\documentclass{article} 
\usepackage{makeidx} 
\makeindex 
\begin{document} 
Hello\index{hello} world\index{world} 
\printindex 
\end{document}
\end{verbatim}

The build file that uses Makeindex as an indexing processor can look like this:

\begin{verbatim}
Make:htlatex {} 
Make:makeindex {} 
Make:htlatex{}
\end{verbatim}

The command \verb| Make:htlatex| compiles the document using \LaTeX\ with 
the \TeX4ht package automatically loaded, \verb|Make:makeindex| calls 
Makeindex and the final \verb| Make:htlatex| compiles the document with 
the index included. Note that instead of page numbers, the numbers in the index
are numbered consecutively for each \verb|\index| command. Due to that,
we can point every index entry back to the original location.

To use a build file, use the \verb|-e| command line option:

\begin{verbatim}
$ make4ht -e build.lua foo.tex
\end{verbatim}

\section{Documentation and server side compilation}

We have made progress in writing the new \TeX4ht documentation. 
In addition to chapters on available configuration commands and command line options,
it contains also a how-to guide with common tasks and developer information for package writers.
It is available at the following address:

\smallskip
\noindent\tbsurl{www.kodymirus.cz/tex4ht-doc}
\smallskip

\noindent It also shows an important development, the usage of server-side compilation. 
Thanks to Github Actions, documentation is automatically generated from
\LaTeX\ sources every time we update them. We don't need to upload generated 
HTML files to a web server, everything is handled automatically by Github Actions.
In the background, the Docker container for \TeX\ Live is used. It enables us
to call any command available in \TEX\ Live, including \makefourht. 
A similar service is also provided by GitLab and other source code hosting platforms.

This method was also used for the conversion of Overleaf projects linked to Github
repositories, the HTML version of \makefourht\ documentation, or even a simple blog:

\smallskip
\noindent\tbsurl{www.kodymirus.cz/testblog/}



\section{JATS format support}

We recently added a support for the JATS XML format, which is intended for 
scientific article authoring. It is an important development, as it is required
by many publishers for the article archiving or for the further processing.

It is also the first output format for \TeX4ht that I personally created. 
The specification is quite strict on the structure of the document, 
which is often inconsistent with the free document structure used in \LaTeX. 
\makefourht\ postprocessing using LuaXML is heavily used to produce the 
correct structure. 

The support is still fairly basic, so user feedback and bug reports are appreciated.

\section{MathJax configuration}

We continue to extend support for MathJax, which can be used to render
math in converted documents. The resulting document typically looks
much better than documents converted using the default \TeX4ht method,
which uses a mix of images and HTML formatting. It is also better
for accessibility, as MathJax can support screen readers, for example.

One downside is that MathJax does not support custom commands out of the box. 
It needs a special configuration that declares these commands. An example
of a configuration file that provides that could be the following one:

\begin{verbatim}
\Preamble{xhtml,mathjax} 
\Configure{MathJaxConfig}{{ 
tex: { 
 \detokenize{% 
  macros: { 
    foo: "\\mathrm{foo}", 
  } 
 } 
} 
}} 
\begin{document} 
\EndPreamble
\end{verbatim}

The \verb|\Configure{MathJaxConfig}| command contains JavaScript
code that configures MathJax. The \texttt{macros} table, which needs
to be located inside the table \texttt{tex}, can contain user macros.
The \verb|\detokenize| command is used to prevent problems with backslash
characters, which need to be doubled. In the example, we define the
\verb|\foo| macro, which prints the word ``foo'' in roman font.


\bibliographystyle{tugboat}
%\nocite{book-minimal}      % make the bibliography non-empty
\bibliography{tugboat}

\makesignature
\end{document}
