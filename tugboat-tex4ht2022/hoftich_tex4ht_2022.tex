% This template file is public domain.
%
% TUGboat class documentation is at:
%   http://mirrors.ctan.org/macros/latex/contrib/tugboat/ltubguid.pdf
% or
%   texdoc tugboat

\documentclass{ltugboat}

\usepackage{microtype}
\usepackage{graphicx}
\usepackage[hidelinks,pdfa]{hyperref}

%%% Start of metadata %%%

\title{What's new in \TeX4ht}

% repeat info for each author; comment out items that don't apply.
\author{Michal Hoftich}
\address{Magdaleny Rettigové 4 \\ Prague, 110 00\\ Czechia}
\netaddress{michal.h21@gmail.com}
\personalURL{https://www.kodymirus.cz/}
%\ORCID{0}
% To receive a physical copy of the TUGboat issue, please include the
% mailing address we should use, as a comment if you prefer it not be printed.

% The preferred mailing address is: Bellušova 1829, Prague 5, 155 00

%%% End of metadata %%%

\begin{document}
\maketitle

\begin{abstract}
Please write an abstract. Please use only standard \LaTeX\ and \TUB\
macros in the abstract, not article-specific abbreviations or other
macros; that helps us make the web pages. Please avoid bibliography
references in the abstract.
\end{abstract}

\section{Introduction}

This is an example article for \TUB, linked from
\url{https://tug.org/TUGboat/location.html}.

\section{Basic packages and hyperlinks}

The standard \texttt{graphicx} and \texttt{xcolor} packages work fine
for image and color handling, respectively, as does the
\texttt{hyperref} package for active urls. \TUB\ is produced using
\acro{PDF} files exclusively, but conversion from \DVI\ is fine and supported.

Please do \emph{not} hide hyperlinks via \cs{href}, as in:
\begin{verbatim}
\href{https://some/url}{some text} % no!
\end{verbatim}
When printed on physical paper, the url will not be visible, and indeed,
readers will not know it exists. Instead, use something like
\begin{verbatim}
some text (\tbsurl{some/url}) % yes
\end{verbatim}
so the url will be in the typeset output.

The \verb|\tbsurl{some/url}| \TUB\ macro expands to
\verb|\href{https://some/url}{some/url}|, so that the url will be both
visible and active, yet need be written only once in the source. There
is also \cs{tbhurl} to use \texttt{http://}, when \texttt{https} is not
supported. We generally prefer to omit the network protocol when it is
\texttt{http} or \texttt{https}, for brevity, but include it in other
cases.

\section{Abbreviation macros}

The \texttt{ltugboat} class provides many abbreviation commands; here
are a few of the most common:

% verbatim blocks are often better in \small
\begin{verbatim}[\small]
\AllTeX \AMS
\BibTeX \Cplusplus \CTAN
\DVI
\HTML \LaTeXe
\MacOSX \MathML \MF
\PDF \PS
\TUB \TUG \tug
\WEB
\XeLaTeX \XeTeX \XML

\Dash \slash

\acro{FRED} -> {\small[er] FRED}  % please use!
\cs{fred}   -> \fred
\meta{fred} -> <fred>
\booktitle{Book of Fred}
\end{verbatim}

\section{Figures}

The standard \texttt{figure} environment produces a column-width figure;
this is desirable when at all possible. The \texttt{figure*} environment
produces a full-width (across both columns) figure when needed.
Analogously for \texttt{table} and \texttt{table*}. 

Please put captions below figures, but above tables. That's our
convention.

Don't worry overmuch about figure placement, as this will likely change
with any edits.

\begin{figure}
A column-width figure, made with \cs{begin}\tubbraced{figure}.
\caption{Caption for column-width figure.}
\label{fig.colwidth}
\end{figure}

\begin{figure*}
A full-width figure, made with \cs{begin}\tubbraced{figure*}.
\caption{Caption for full-width figure.}
\label{fig.fullwidth}
\end{figure*}

\section{Bibliographies}

For references to other issues of \TUB, please use the format
\textsl{volno:issno}, e.g., ``\TUB\ 32:1'' for volume~32, number~1.

The \TUB\ style documentation is the \texttt{ltubguid} document at
\tbsurl{ctan.org/pkg/tugboat}. For general \CTAN\ package references, we
recommend the form just shown, using \texttt{/pkg/}; if you need to
refer to a specific file location, use
\texttt{http://mirror.ctan.org/\textsl{path}}.

We recommend using \BibTeX\ (but don't require it), with the
\texttt{tugboat} \BibTeX\ style. The \texttt{biblio} directory on \CTAN,
\tbsurl{ctan.org/pkg/biblio}, provides files \texttt{tugboat.bib} with
an essentially complete bibliography of \TUB, and \texttt{texbook3.bib}
(and others there) with many \TeX-related citations. Both of these (and
many more) should also already be in a non-minimal \TeX\ distribution.

Email \verb|tugboat@tug.org| with any problems or questions.

\bibliographystyle{tugboat}
\nocite{book-minimal}      % make the bibliography non-empty
\bibliography{xampl}       % xampl.bib comes with BibTeX

\makesignature
\end{document}
