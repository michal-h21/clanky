\documentclass{beamer}
\usetheme{metropolis}           % Use metropolis theme
\title{\TeX4ht configurations}
\date{\today}
\author{Michal Hoftich}
\institute{Charles University, Faculty of Education, Library}
\begin{document}
  \maketitle
  \begin{frame}
  \tableofcontents
  \end{frame}
  \section{Introduction}
  \begin{frame}
   \frametitle{Basic Information}
    \begin{itemize}
      \item \TeX4ht converts \LaTeX\ files to HTML, ODT, Docbook, Tei and other formats
      \item It uses \LaTeX\ itself for the conversion, so it supports custom commands and most 
        packages.
      \item It supports basic features, like font sizes, weights and style out of the box, but if
        we want to support more advanced features, like sections, footnotes, tables or images, 
        we must provide configuration files.
      \item As any \LaTeX\ package can redefine \LaTeX\ core or other package's code, conflicts can 
        happen.
    \end{itemize}

  \end{frame}
  \section{Process Overview}
  \begin{frame}[fragile]
   \frametitle{\LaTeX to DVI}
    \begin{itemize}
      \item \TeX4ht code is loaded using the \verb|tex4ht.sty| package. 
      \item It loads all configuration files
        for the supported packages and inserts XML tags. 
      \item Tags are passed in \verb|\special| commands to the DVI file.
    \end{itemize}
  \end{frame}
  \begin{frame}[fragile]
   \frametitle{The \texttt{tex4ht} command}
    \begin{itemize}
      \item The DVI file is processed by the \verb|tex4ht| command, that produces XHML or HTML files and 
        two special files:
        \begin{itemize}
      \item The \verb|.lg| file contains instructions for CSS or generated picture file names.
      \item The \verb|.idv| file is special version of the DVI file that contains only parts of the original DVI file 
        that should be transformed to pictures.
    \end{itemize}
    \end{itemize}
  \end{frame}
  \begin{frame}[fragile]
    \frametitle{The \texttt{t4ht} command}
   \begin{itemize}
     \item It creates the CSS file and creates pictures.
   \end{itemize}
  \end{frame}
  \section{make4ht}
  \begin{frame}
    \frametitle{Conversion scripts}
    \begin{itemize}
      \item \TeX4ht provided several scripts to make the conversion easier, but they were obsoleted by two tools:
        \begin{itemize}
      \item tex4ebook converts \LaTeX\ to Epub and Mobi formats.
      \item make4ht converts to all other supported formats.
    \end{itemize}
  \item both have similar features and options.

    \end{itemize}
  \end{frame}

\begin{frame}
    \frametitle{make4th and tex4ebook features}
    \begin{itemize}
      \item Unicode output by default.
      \item It is possible to select Lua\LaTeX or Xe\LaTeX for the compilation.
      \item Easy selection of the output format.
      \item Configurable level of terminal output.
      \item Extension support.
      \item Post-processing of the generated files.
      \item Lua build file support.
    \end{itemize}
  \end{frame}

\begin{frame}[fragile]
    \frametitle{Basic usage}
    \begin{verbatim}
$ make4ht [options] filename.tex "[tex4ht options]"
  \end{verbatim}
  or 
  \begin{verbatim}
$ tex4ebook [options] filename.tex \
"[tex4ht options]"
  \end{verbatim}
\end{frame}

\begin{frame}[fragile]
  \frametitle{Engine selection}
Compile file using Lua\LaTeX:
\begin{verbatim}
$ make4ht -l filename.tex
\end{verbatim}
Compile file using Xe\LaTeX:
\begin{verbatim}
$ make4ht -x filename.tex
\end{verbatim}
\end{frame}

\begin{frame}[fragile]
\frametitle{Select output format}
Convert to ODT:
\begin{verbatim}
$ make4ht -f odt filename.tex
\end{verbatim}

Convert to HTML 5 (default):

\begin{verbatim}
$ make4ht -f html5 filename.tex
\end{verbatim}

Convert to Epub 3:

\begin{verbatim}
$ tex4ebook -f epub3 filename.tex
\end{verbatim}

\end{frame}

\begin{frame}[fragile]
  \frametitle{Modes}
  make4th compiles \TeX\ file three times by default. To make the compilation faster, use the \texttt{draft} mode:

\begin{verbatim}
$ make4ht -m draft filename.tex
\end{verbatim}

To clean all temporary files, as well as generated XML files and pictures, use:
\begin{verbatim}
$ make4ht -m clean filename.tex
\end{verbatim}
\end{frame}


\begin{frame}[fragile]
\frametitle{Terminal output}
  make4ht hides most of the terminal output by default. It only shows errors and warnings. To show all
  messages, use:
\begin{verbatim}
$ make4ht -a debug filename.tex
\end{verbatim}
\end{frame}

\begin{frame}[fragile]
\frametitle{Extensions}
Extensions can be enabled and disabled using the following syntax:
\begin{verbatim}
# enable extension
$ make4ht -f html5+<extension_name> filename.tex
# disable extension
$ make4ht -f html5-<extension_name> filename.tex
\end{verbatim}
One extension \verb|common_domfilters|, is used by default. It fixes common issues in the generated HTML and XML files.
To disable it, use:

\begin{verbatim}
$ make4ht -f html5-common_domfilters filename.tex
\end{verbatim}

\end{frame}

\begin{frame}[fragile]
  \frametitle{Extensions (continued)}

To compile RTeX documents (you need Knitr installed):

\begin{verbatim}
$ make4ht -f html5+preprocess_input filename.rtex
\end{verbatim}


List of supported extensions can be found in the make4ht documentation.


\end{frame}

\begin{frame}[fragile]
\frametitle{Build files}
\begin{itemize}
  \item Build files can be used to change the compilation sequence.
  \item They can define post-processing filters (Lua functions on external commands).
  \item It is also possible to specify command for the picture generation (like \texttt{dvisvgm} or \texttt{dvipng}).
\end{itemize}
\end{frame}

\begin{frame}[fragile]
  \frametitle{Sample build file}
\begin{verbatim}
-- support the fast draft mode
if mode=="draft" then
  Make:htlatex{}
else
  Make:htlatex{}
  -- create index page using the xindex command
  Make:xindex {}
  Make:htlatex{}
  Make:htlatex{}
end
\end{verbatim}
\end{frame}

\begin{frame}[fragile]
  \frametitle{Use build file}
  \begin{itemize}
    \item The previous sample file can be used to support index in the HTML file.
    \item The \texttt{mode} variable holds value of the \texttt{-m} option.
    \item Commands are executed using \verb|Make:command_name{options}|.
  \end{itemize}

  To pass the build file to make4ht use:

\begin{verbatim}
$ make4ht -e build.lua filename.tex
\end{verbatim}
\end{frame}

\section{\TeX4ht configuration}

\begin{frame}[fragile]
  \frametitle{\TeX4ht options}
\TeX4ht is highly configurable. The easiest way how to change it's features is using options. They can be 
required using:
\begin{verbatim}
$ make4ht filename.tex "option1,option2,..."
\end{verbatim}

\end{frame}

\begin{frame}
  \frametitle{Useful options}
Each output format has different set of options. Some useful options for HTML include:

\begin{description}
  \item[fn-in] put footnotes at the bottom of the current page instead of standalone pages.
  \item[svg] create SVG pictures.
  \item[2] put each section to a separate HTML page.
  \item[sec-filenames] name HTML files for sections according to the section title.
  \item[info] output description of the basic configurations to the .log file.
\end{description}

\end{frame}

\begin{frame}[fragile]
  \frametitle{Configuration files}
  \begin{itemize}
    \item Configuration files can pass the options.
    \item They can contain the CSS code.
    \item They can be used to configure the XML tags.
  \end{itemize}
  They can be required using:
\begin{verbatim}
$ make4ht -c config.cfg filename.tex
\end{verbatim}
\end{frame}

\begin{frame}[fragile]
  \frametitle{Basic structure of the configuration file}
\begin{verbatim}
\Preamble{xhtml,options}
% early configurations
\begin{document}
% configurations that should appear 
% after \begin{document}
\EndPreamble
\end{verbatim}
\begin{itemize}
  \item \verb|\Preamble| can pass the options. \texttt{xhtml} option needs to be alway first, even if no other option is used.
  \item \TeX4ht redefines all commands after the document preamble, but some code may be executed later. This can be overriden
    by configurations put after the \verb|\begin{document}|.
  \end{itemize}
\end{frame}
\begin{frame}[fragile]
  \frametitle{Sample configuration file}
The following configuration file can be used to create HTML files
where sections are in separate files named after the secton titles,
with footnotes at the end of the each file where they appear.

CSS is used to set the maximum width of the page.

\begin{verbatim}
\Preamble{xhtml,2,sec-filename,fn-in}
\Css{body{max-width:60ch;margin:1em auto;}}
\begin{document}
\EndPreamble
\end{verbatim}
\end{frame}

\begin{frame}[fragile]
  \frametitle{Require external CSS file}
In this example we use the LaTeX.css style to change the document appearance completely:

\begin{verbatim}
\Preamble{xhtml,2,sec-filename,fn-in}
\Configure{@HEAD}{
  \HCode{
    <link rel="stylesheet" 
    href="https://latex.now.sh/style.css">
    \Hnewline
  }
}
\begin{document}
\EndPreamble
\end{verbatim}

\end{frame}


\begin{frame}[fragile]
  \frametitle{Commands useful in the configuration files}
\begin{itemize}
  \item The \verb|\Configure| command is used to insert code to hooks that were declared 
    by configuration files for used packages and \LaTeX\ core.
  \item The \verb|\HCode| command is used to insert XML tags.
  \item The \verb|\Css| command is used for shorter CSS code.
  \item The \verb|\Hnewline| command is used to output a line break to the XML file.
\end{itemize}
\end{frame}

\begin{frame}[fragile]
  \frametitle{Math appearance}
  By default, math output is mixture of HTML elements and pictures. It doesn't look well.
  It is better to use the SVG graphics for pictures and require pictorial math also for inline math:
\begin{verbatim}
$ make4ht filename.tex "pic-m,svg"
\end{verbatim}
\end{frame}

\begin{frame}[fragile]
  \frametitle{Faster SVG math}
You can use the \verb|dvisvgm_hashes| extension for faster compilation of documents that produce
lot of images. It can use multiple processes and compiles only pictures that changed from the 
previous run:

\begin{verbatim}
$ make4ht -f html5+dvisvgm_hashes filename.tex \ 
"svg,pic-m"
\end{verbatim}
\end{frame}

\begin{frame}[fragile]
  \frametitle{MathML}
  MathML is a better solution than pictures in general. The downside is that it doesn't work in all web browsers:
\begin{verbatim}
make4ht filename.tex "mathml"
\end{verbatim}

You can use the MathJax library to require MatML support in all browsers:

\begin{verbatim}
make4ht filename.tex "mathml,mathjax"
\end{verbatim}
\end{frame}


\begin{frame}[fragile]
\frametitle{MathJax output}
You can also let MathJax to completely handle math:

\begin{verbatim}
make4ht filename.tex "mathjax"
\end{verbatim}

\TeX4ht doesn't expand math contents in this case, so you will need to pass definitions of custom commands to MathJax.
\end{frame}

\begin{frame}[fragile]
  \frametitle{Pass configuration to MathJax}

\begin{verbatim}
\Preamble{xhtml,mathjax}
\Configure{MathJaxConfig}{{
    tex: {
      \detokenize{%
      macros: {
        mycmd: "\\sin{a}"
      }
  }
}
}}
\begin{document}
\EndPreamble
\end{verbatim}
\end{frame}

\begin{frame}[fragile]
  \frametitle{Images}
  \TeX4ht needs to know image dimensions to correctly set them in the HTML file. It uses a .xbb file for that. It can be created using:

\begin{verbatim}
ebb -x imagename.jpg
\end{verbatim}
\end{frame}

\section{Troubleshooting}
\begin{frame}[fragile]
\frametitle{Package conflicts}
\begin{itemize}
  \item \TeX4ht needs to redefine lot of internal \LaTeX\ commands, as well as commands from various packages. It may sometimes lead to 
    package clashes, if other packages redefine the same commands.
  \item Some packages don't work well with \TeX4ht, because they require the PDF mode for example.
  \item It is ideal to run \TeX4ht after you add any new package, to find if it fails.
  \item Please let us know about conflicts at \TeX4ht mailing list, issue tracker, or at TeX.sx (use the \textit{tex4ht} tag).
\end{itemize}
\end{frame}
\begin{frame}[fragile]
  \frametitle{First aid for the package conflict}
  We want to avoid need to any changes in your \TeX\ documents for the \TeX4ht support, but it is sometimes the easiest way how to 
  handle the package conflict.

Use the \verb|\ifdefined\HCode| command to detect \TeX4ht, and avoid loading of the conflicting packages using that:

\begin{verbatim}
\documentclass{article}
...
\ifdefined\HCode
\usepackage{graphicx}
\else
\usepackage[pdftex]{graphicx}
\fi
...
\end{verbatim}

\end{frame}


\begin{frame}
   \frametitle{Links}
     \TeX4ht homepage \\ \url{tug.org/tex4ht/}\\
     New \TeX4ht documentation \\ \url{www.kodymirus.cz/tex4ht-doc/tex4ht-doc.html}\\
     \TeX4ht options\\
     \url{https://www.kodymirus.cz/tex4ht-doc/texfourhtOptions.html}\\
     Make4ht documentation \\ \url{www.kodymirus.cz/make4ht/make4ht-doc.html}\\
     Issue tracker\\
     \url{puszcza.gnu.org.ua/bugs/?group=tex4ht}\\

\end{frame}
  % \begin{frame}[]{}
  % \end{frame}
\end{document}
