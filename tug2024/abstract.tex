\documentclass[]{article}
\begin{document}

\begin{abstract}
    

In my presentation, I will introduce a new tool, Rmodepdf, for converting web pages to PDF using Lua\LaTeX. It supports a technique known as the ``reader mode,'' which web browsers employ to display clean text on pages devoid of navigation elements or ads. The primary purpose of Rmodepdf is to facilitate the reading of longer articles in e-book readers or for web archiving purposes.

It uses a new HTML parser in the LuaXML package for HTML processing. We will also demonstrate two packages used in the conversion template, which enable automatic typesetting for different output devices without the user's intervention. The first is the Responsive package, which brings responsive design methods to \LaTeX, and enables the setting of font sizes or line heights that match the output page size. The second is the Linebreaker package, which prevents line overflow and thus enables fully automatic document typesetting.
\end{abstract}

\end{document}
