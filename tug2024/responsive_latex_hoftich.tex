%! TeX program = lualatex
\ifdefined\directlua
\RequirePackage{luatex85}
\fi

\documentclass{ltugboat}
% \usepackage[T1]{fontenc}
% \usepackage[utf8]{inputenc}
\usepackage{fontspec}
\setmainfont[Ligatures={TeX,Rare}]{Latin Modern Roman}
\usepackage[czech,english]{babel}
\usepackage{luavlna}
\usepackage[noautomatic]{responsive}
\usepackage{lua-widow-control}
\usepackage{csquotes}
\usepackage[]{linebreaker}
\usepackage{amsmath,amsfonts}

\usepackage{graphicx}
\usepackage{caption}
\usepackage{subcaption}
\usepackage{lipsum}


\linebreakersetup{
%   maxtolerance =90 ,
  maxemergencystretch=1em,
%   maxcycles=4
}

% \usepackage[
%   backend=biber,
%   % style=iso-numeric,
%   style=numeric,
%   sortlocale=cs,
%   autolang=other,
%   bibencoding=UTF8,
%   mincitenames=2,
%   maxcitenames=2,
% ]{biblatex}
% \addbibresource{responsive.bib}
\usepackage[
  implicit=false,
  hidelinks,
]{hyperref}



% pro případ, kdy je \lipsum moc velké 
\newcommand\smalllipsum{%
Lorem ipsum dolor sit amet,
consectetuer adipiscing elit.
Ut purus elit, vestibulum ut,
placerat ac, adipiscing vitae,
felis. Curabitur dictum gravida 
mauris. Nam arcu libero, nonummy 
eget, consectetuer id, vulputate 
a, magna. Donec vehicula augue eu neque.
}

\newcommand\printsize[1]{\csname #1\endcsname\par\noindent Sample\par}
\newcommand\showscale[2][.5\textwidth]{%
      % \setsizes[38]{25}
      \printsize{huge}
      \printsize{LARGE}
      \printsize{Large}
      \printsize{large}
      \hrule
      \printsize{normalsize}
      \hrule
      \printsize{small}
      \printsize{footnotesize}
}


\title{Web page to PDF conversion with Rmodepdf: Leveraging Lua\LaTeX\ for e-book reader-friendly documents}

% repeat info for each author; comment out items that don't apply.
\author{Michal Hoftich}
\address{Magdalény Rettigové 4\\ Praha, 116 39\\ Czechia}
\netaddress{michal.h21 (at) gmail dot com}
\personalURL{https://www.kodymirus.cz/}
\begin{document}

\maketitle

\begin{abstract}
This article presents the use of responsive design methods and advanced features
of Lua\LaTeX\ for automatic document typesetting intended for various target outputs,
both printed and electronic, such as mobile phones, tablets, or e-readers.

Specifically, it focuses on the use of Lua\LaTeX\ for automated
typesetting with the help of the \tbcode{Responsive} package~\cite{responsive}
for setting font size and line spacing according to page size, the
\tbcode{Luavlna} package~\cite{luavlna} to prevent single-character
prepositions at the ends of lines, the \tbcode{Lua-widow-control}
package~\cite{lua-widow-control} to minimize widows and orphans at the ends and
beginnings of pages, and the \tbcode{Linebreaker} package~\cite{linebreaker} to
prevent line overflow.

\end{abstract}


\section{Introduction}

Some time ago, I acquired an e-book reader, but I still read most texts on my
PC screen because they come from web sources. It occurred to me that I could
save longer articles for later reading on my e-reader. There are, of course,
several applications for this purpose, but I decided to create my own, tailored
exactly to my needs and preferences. Another motivation is the opportunity to
learn something new and create packages that will be useful for other \TeX\
users as well.

My goal is to make the solution as automated as possible, so I don't have to
deal with overflow lines or other errors that would require manual
intervention. Thanks to the capabilities of Lua\LaTeX, such a solution is
already possible, as we will demonstrate in the following text.

Because Lua\TeX\ includes the Lua programming language in \TeX\ distributions,
I used it to create my project \tbcode{Rmodepdf}~\cite{rmodepdf}. It uses the
\tbcode{LuaXML} package~\cite{luaxml} to transform HTML into \TeX{} and few
externall commands -- \tbcode{Curl} for downloading of the web pages, 
and \tbcode{Rdrview} \cite{rdrview}, which removes navigation
elements, advertisements, and other distractions from the page.
The \tbcode{Rdrview} is based on the JavaScript library used by the
\tbcode{Firefox} browser for its Reader Mode. However, it has been translated
from JavaScript to C, making it significantly faster and eliminating the need
for any additional dependencies.

During the development of the program, I also created or significantly expanded
three \LaTeX\ packages that may be useful on their own. For the \tbcode{LuaXML}
package, I created an HTML parser that allows web pages to be processed
directly from the Lua language. The \tbcode{Responsive} package enables the
creation of templates that adjust font size, page margins, and other parameters
according to the current page size. Finally, the \tbcode{Linebreaker} package
prevents line overflow, which is crucial in automated document typesetting
where we neither want nor can manually correct such errors.

In addition to saving articles for reading on an e-reader, there are other ways
to utilize the \tbcode{Rmodepdf} program. One such use is archiving web
content on paper. By removing all navigation elements, we obtain a document
that can be easily printed, bound, and archived as a book.

\section{Basic usage of \tbcode{Rmodepdf}}

The \tbcode{Rmodepdf} accepts one or more URLs as arguments. It is also
possible to use the addresses of local HTML files.

\begin{verbatim}
$ rmodepdf <url1> <url2>
\end{verbatim}


The output file is named based on the title of the first document. If the title
cannot be found, a name based on the current date and time is chosen. The
generated name is displayed in the terminal output. You can specify a custom
file name using the \texttt{-o} option.

If you prefer not to compile the document directly but only to display the text
generated by the \LaTeX\ document, you can use the \texttt{-p} option.

\begin{verbatim}
$ rmodepdf -p <url> > foo.tex
\end{verbatim}

You can choose a different page size using the \texttt{-P} option. By default,
the page size and margins are set for e-book readers, but you can also select
other sizes, such as A4 paper size. The page style is currently set to empty
(\textit{blank}), but you can change it using the \texttt{-s} option.

\begin{verbatim}
$ rmodepdf -P a4paper -s plain <url>
\end{verbatim}

To enhance speed, images are stored in a local directory. By default, this is
the \texttt{img/} subdirectory within the current directory, but you can
specify a different directory using the \texttt{-i} option. 

\begin{verbatim}
# save the document as foo.pdf and 
# save images in the temp dir
$ rmodepdf -o foo.pdf -i /tmp/img <url>
\end{verbatim}

You can disable image downloading entirely with the \texttt{-n} option. 
The \tbcode{Rmodepdf} also detects and displays \LaTeX\ mathematical commands
embedded in web pages that use \tbcode{MathJax} or \tbcode{KaTeX} for
rendering. This default behavior can be disabled using the \texttt{-N} option.
Additionally, the removal of page elements using \tbcode{Rdrview} can be
disabled with the \texttt{-R} option.




\section{Configuration}

\subsection{Settings}

It often happens that during conversion you encounter errors or wish to change
how certain elements on the page are converted into \LaTeX. Therefore,
\tbcode{Rmodepdf} provides the option to use a Lua configuration script. This
script allows you to modify the code of translated pages, define conversion
rules, set variables, or change templates as needed. The configuration file is
loaded using the option \texttt{-c}.

\begin{verbatim}
$ rmodepdf -c script.lua <url>
\end{verbatim}

The script might look like this, for example:

\begin{verbatim}
add_to_config {
  document = {
    preamble_extras = [[
    \setmainfont{Linux Libertine O}
    ]],
  },
  img_convert = {
    -- modify the command used for 
    -- conversion of SVG images to PDF
    svg = "cairosvg -o ${dest} -",
  },
}
\end{verbatim}


This example uses the command \texttt{add\allowbreak\_to\allowbreak\_config}, which safely copies new
configuration values into the original configuration. If you only want to set a
single configuration value, you can also directly write to the \texttt{config}
table:

\begin{verbatim}
config.document.geometry = "a6paper"
\end{verbatim}

The \texttt{config} table contains several subtables that you can configure.
The \texttt{document} subtable includes properties of the output document, such
as \texttt{preamble\_extras} for adding additional code to the document
preamble, or \texttt{geometry}, which allows you to directly specify the
dimensions of the page or margins of the output document.

The subtable \texttt{img\_convert} defines commands for converting image formats
used on the converted web pages that are not supported in Lua\LaTeX\ to one of
the supported formats. For example, in the sample, we define a command to
convert from SVG format to PDF. This command must support reading from standard
input, and you can specify the output file name using the template \verb|${dest}|.

The subtable \texttt{html\_latex} contains settings for translating \LaTeX\ code
embedded in web pages. The \texttt{ignored} item contains a list of HTML
elements where embedded \LaTeX\ code should not be searched for. Typically, this
includes elements like \verb|<pre>|, which contain source code that should
not be processed in our document.

The subtable \texttt{pages} contains converted files and their metadata. Its
content is populated after the configuration script runs, so it is not
available beforehand but is utilized in templates. It includes items such as
\texttt{language} for the document language, \texttt{title} for the document
title, and \texttt{content} which contains the \LaTeX\ code of the document for
transformation.

\subsection{Callbacks}


The configuration script is executed before the actual conversion, so it cannot
directly influence the conversion process. However, we can define several
callback functions that allow us to affect the conversion. These functions are
as follows:

\begin{description}
  \item[\texttt{preprocess\_content}] modify string with the raw HTML before
    readability and DOM parsing.
  \item[\texttt{preprocess\_dom}] modify DOM object before fetchching of images
    or handling of MathJax.
  \item[\texttt{postprocess\_dom}] modify DOM after all processing by Rmodepdf.
  \item[\texttt{postprocess}] late post-processing of the config table.
\end{description}

The most useful are the first three. The \texttt{preprocess\_content} function
takes a string parameter with the HTML code of the page as it was downloaded
from the original website, before any modifications by \tbcode{Rdrview}. Here,
you can use Lua string functions to fix certain elements that may cause issues
during processing with \tbcode{Rdrview}. This method is quite limited and,
especially when using regular expressions, it can cause more harm than good.
Therefore, use it with caution.

The difference between the next two callbacks is that with the first one, you
can still influence image downloading or the processing of \LaTeX\ commands. For
modifications to the final version of the document, it is best to use
\texttt{postprocess\_dom}.

Both functions receive a \tbcode{LuaXML} DOM object as a parameter. This
allows you to safely traverse and transform the entire document.
\tbcode{LuaXML} includes many functions for working with the DOM; here, we
will introduce just a few basics. For example, the following example prints the
resulting DOM object as HTML code:

\begin{verbatim}
function postprocess_dom(dom)
  print(dom:serialize())
  return dom
end
\end{verbatim}

The \texttt{dom:serialize()} method to obtains the HTML
code from the DOM object, which we then print using the \texttt{print} command.
It is important to return the DOM at the end of the function, this ensures that
any modifications made to the DOM are preserved and applied to the final
document.



Here's a slightly more complex example. Let's assume that \tbcode{Rdrview} did
not remove a menu that might look like this:

\begin{verbatim}
<div class="menu">
... menu contents ...
</div>
\end{verbatim}

We can use the \texttt{postprocess\_dom} function to remove this menu:

\begin{verbatim}
function postprocess_dom(dom)
  -- Find the menu using a CSS selector
  local menu = dom:query_selector(".menu")

  -- Iterate over the menu elements 
  -- and remove each one
  for _, el in ipairs(menu) do
    el:remove_node()
  end
  
  -- Return the modified DOM
  return dom
end
\end{verbatim}

In this example:

\begin{enumerate}
  \item We use the \texttt{query\_selector} method to find all elements with
    the class \texttt{menu}.
  \item Iterate over each element retrieved in the previous step
    using a \texttt{for} loop.
  \item Remove each menu element using the \texttt{remove\_node} method.
  \item Return the modified DOM at the end of the function.
\end{enumerate}

This ensures that any remaining menus are removed from the final document.

\subsection{Transformation from HTML to \LaTeX}

We perform the conversion of HTML elements to \LaTeX\ using the
\texttt{luaxml-transform} library. This library allows us to declare simple
rules for transforming XML or HTML elements into text. Elements can be selected
using CSS selectors, which is important because elements with the same name but
different classes may need to be converted differently. For example,
\verb|<span>| or \verb|<div>| elements are often used as universal tags, but
their intended display can vary greatly, depending on their class.

In the configuration file, the variable \texttt{htmlprocess} contains an object with
rules for converting HTML elements. It provides two main functions:
\texttt{htmlprocess.reset\_actions}, which clears all rules for a given selector, and
\texttt{htmlprocess.add\_action}, which adds new rules.
The following code displays some basic usage of the transformation library:

\begin{verbatim}
htmlprocess.reset_actions("br")
htmlprocess.reset_actions("figure")
htmlprocess.add_action("br", "\n\n")
htmlprocess.add_action("img", 
[[\includegraphics[max width=\textwidth]{@{src}}]])
htmlprocess.add_action("figure", 
"\n\n\\medskip\n\n\\noindent %s")
\end{verbatim}

In this example, we reset the default rules for the \verb|<br>| and
\verb|<figure>| elements and introduce custom rules with specific syntax. The
rules adhere to the following conventions:

\begin{itemize}
  \item The \verb|%s| string inserts the transformed content of the element.
    It is crucial to include \verb|%s| in most rules to ensure the content is
    correctly processed; omitting it would hide the entire element's content.
    Since the \verb|<br>| element does not contain any text, it is unnecessary
    to use \verb|%s| with it.
  \item In the rule for the \verb|<img>| element, \verb|@{src} | inserts the
    value of the src attribute, which contains the image's address.
\end{itemize}

The following example demonstrates the use of CSS selectors for classes and
attribute value comparisons to handle different types of links differently:

\begin{verbatim}
htmlprocess.reset_actions("a")
htmlprocess.add_action("a.easy-footnote-to-top", "")
htmlprocess.add_action('a[href|="#easy-footnote"]', "%s")
\end{verbatim}

Links with the class \verb|a.easy-footnote-to-top| are hidden because the
replacement text string is empty. However, for links whose \verb|href|
attribute starts with \verb|'#easy-footnote'|, only their text content is
displayed.

This was just a brief introduction to the transformation possibilities using
\texttt{luaxml-transform}. You can find many more examples in the
\tbcode{LuaXML} manual.





\section{Template}

After converting from HTML to \LaTeX, we need to combine the resulting code into
a single document that can be compiled. Therefore, \tbcode{Rmodepdf} includes a
simple templating system that allows us to merge individual pages and their
metadata together.

A basic template might look like this:

\begin{verbatim}
\documentclass{article}
\usepackage{linebreaker,responsive}
\usepackage[_{document.languages}%s/{,}]
{babel}
\usepackage[@{document.geometry}]{geometry}
\pagestyle{@{document.pagestyle}}
@{document.preamble_extras}
\begin{document}
_{pages}
\selectlanguage{@{language}}
?{title}{Title: @{title}}\par}{}
?{author}{Author: @{author}\par}{}
\href{@{url}}{@{url}}\par
@{content}
/{\clearpage}
\end{document}
\end{verbatim}


Templates contain three syntactic constructs. The basic one is printing a
variable using \verb|@{variablename}|. Variables are contained in the
\verb|config| table, and using a dot, we can also print properties of
sub-tables. For example, \texttt{@\{document.\allowbreak preamble\_extras\}} prints the
\texttt{config.\allowbreak document.preamble\_extras} variable.

The next construct is loops. They have the syntax 
\verb|_{variablename}loop code/{separator}|. 
Variables that can be used must be arrays. For example,
\verb|document.languages| contains the languages of all translated documents in
a format suitable for the Babel package, or \verb|pages|, which contains all
converted documents. In the loop code, variables of the currently processed
array are available. If the array contains only strings, we can use the
placeholder \verb|%s|. This is used for \verb|document.languages|. If the
current object is a table, we can access its fields directly using
\verb|@{variablename}|.

The last construct is conditions. Their syntax is
\verb|?{variablename}{true}{false}|. In the example, we use them to insert the
title and author, because not all pages have these items.

Custom templates can be required using the \verb|-t| option.

\begin{verbatim}
$ rmodepdf -t mytemplate.tex <url>
\end{verbatim}

\section{Automatic Typesetting}

This brings us to the next part. Since \verb|Rmodepdf| compiles web pages
directly into PDF, we cannot easily intervene in the conversion process.
Therefore, the conversion needs to be as automated and error-free as possible.
The output PDF can also have various page sizes. The default size is adapted
for e-book readers, but we might also want to create a PDF suitable for
smartphones or, conversely, a standard A4 size. For all these sizes, we need to
choose different font sizes or page margins. This can be achieved using two new
packages, which we will demonstrate in this section.


\subsection{Responsive Design}

One of the issues that need to be addressed is setting the correct font size
for readability. The default font size in \LaTeX\ is 10 points, regardless of
the page size. This is a suitable font size for an A5 page. For A4 format, the
font size should be larger, and for smaller screens of e-readers and mobile
phones, it can be smaller. Similarly, we can change the line spacing, which
also affects text readability depending on the font size and page size.

Web browsers face a similar problem, as they need to display text on large PC
monitors as well as on smaller screens of laptops, tablets, and mobile phones.
The solution they use is called \textit{responsive design}.

Responsive design is a way of designing web pages that allows flexible and
dynamic adaptation of the appearance and layout of the page content to
different display devices. One of the key elements of responsive design is a
flexible structure that allows elements on the page to be resized to fit the
display device.

Another important element is \textit{media queries}. These allow defining rules
that apply based on the properties of the display device, such as screen width
and height or the type of output (paper, display). Thanks to these rules, the
same page code can be well displayed on both large monitors and mobile devices
or when printed. 



The \tbcode{Responsive} package~\cite{responsive} is inspired by these
principles. Its main function is to set the font size according to the page
size and the approximate number of characters that should fit on the page. It
also sets the typographic scale (affecting font sizes for headings or
footnotes), the font baseline, and supports a simple version of media queries.

\subsubsection{Setting up the \tbcode{Responsive} Package}

\tbcode{Responsive} automatically sets the font size, line spacing, and
typographic scale at the beginning of the document. Default values can be
changed using package options (\cs{usepackage}\texttt{[<options>]\{responsive\}}) or
the \cs{ResponsiveSetup}\texttt{\{<options>\}} command. The \cs{ResponsiveSetup}
command can also be used directly in the document text, for example, for local
font settings changes.

The \tbcode{Responsive} package offers the following options:

\begin{description}
  \item[noautomatic] – prevents automatic setting of font size and line spacing
    at the beginning of the document.
  \item[characters] – number of characters for automatic font size setting.
  \item[scale] – typographic scale used for font sizes.
  \item[lineratio] – ratio used in line spacing calculation.
\end{description}

\subsubsection{Basic Font Size}

The font size can be set using the \cs{setsizes} command. 
\tbcode{Responsive} tries to set the font size so that the desired number of
characters fits on a line on average. The actual number of characters depends
on the text used, as each letter has a
different width when using proportional fonts. In reality, the number of
characters displayed on a line may be slightly higher.

If the number of characters is not specified in the \cs{setsizes} command,
the value of the \texttt{characters} option is used. The following example uses
this option setting. Figure~\ref{fig:fontsize} shows how the same text can be
displayed differently within the same frame, depending on the settings.

\begin{verbatim}
\begin{minipage}{5cm}
\ResponsiveSetup{characters=55}
\setsizes{}
\lipsum[1]
\end{minipage}
\end{verbatim}

\begin{figure*}[htbp]
  \centering 
  \begin{subfigure}[t]{0.45\textwidth}
  \centering 
\fbox{%
\begin{minipage}{5cm}
\ResponsiveSetup{characters=55}
\setsizes{}

\lipsum[1]

\end{minipage}}
\caption{\texttt{characters=55}}
\end{subfigure}
\hfill
\begin{subfigure}[t]{0.45\textwidth}
  \centering 
\fbox{%
\begin{minipage}{5cm}
\ResponsiveSetup{lineratio=38,characters=25}
\setsizes{}

Lorem ipsum dolor sit amet,
consectetuer adipiscing elit.
Ut purus elit, vestibulum ut,
placerat ac, adipiscing vitae,
felis. Curabitur dictum gravida 
mauris. Nam arcu libero, nonummy 
eget, consectetuer id, vulputate 
a, magna. Donec vehicula augue eu neque.

\end{minipage}}
\caption{\texttt{characters=25, lineratio=38}}
\end{subfigure}
  \caption{Difference in font size depending on the number of characters per
  line}\label{fig:fontsize}
\end{figure*}

\subsubsection{Line Spacing}

By default, \LaTeX\ sets the line spacing to the font size multiplied by 1.2.
For different fonts and page sizes, a different line spacing may be
appropriate. Similarly, different values may be suitable for the printed and
electronic versions of the document.

I was inspired by Edoardo Cavazza's article~\cite{cavazza} on readability and
added support for setting line spacing based on the ratio of lowercase letter
height and the \texttt{lineratio} variable. This ratio is obtained by the
following calculation:

\[ \text{line spacing} = \frac{1\text{ex}}{\text{lineratio}/100} \]

You can observe the impact of changing the \texttt{lineratio} value in
Figure~\ref{fig:lineratio}. The choice of its optimal value depends on the font
used and the page size. To achieve maximum output readability, it's advisable
to compare the output using different values.

\begin{figure*}[htbp]
  \centering 
  \begin{subfigure}[b]{0.45\textwidth}
  \centering 
\fbox{%
\begin{minipage}{5cm}
\ResponsiveSetup{lineratio=38}
\setsizes{65}

\lipsum[1]

\end{minipage}}
\caption{\texttt{lineratio=38}}
\end{subfigure}
\begin{subfigure}[b]{0.45\textwidth}
  \centering 
\fbox{%
\begin{minipage}{5cm}
\ResponsiveSetup{lineratio=34}
\setsizes{65}

\lipsum[1]

\end{minipage}}
\caption{\texttt{lineratio=34}}
\end{subfigure}
  \caption{Change in line spacing by changing the \texttt{lineratio} value}\label{fig:lineratio}
\end{figure*}

\subsubsection{Media Queries}

Media queries are a technique that allows web developers to dynamically adapt
the appearance and behavior of web pages based on various device properties,
such as screen width and height, device orientation, color support, and many
others. With these conditions, it is possible to create responsive and flexible
web pages that can automatically adjust to different types and sizes of devices
on which they are displayed.

How can this technique be useful for \LaTeX\ package authors? They could, for
example, set the font size, line spacing, and other elements for specific page
dimensions. After the user chooses the page size according to the device for
which they want to compile the document, these elements are set automatically.
The package author can define, for instance, that if the width of the text line
is less than a certain size, fewer characters will be displayed on it than on
longer lines. The result is shown in Figure~\ref{fig:mediaguery}.

This example will display fewer characters per line if the text width is less than 4 cm.

\begin{verbatim}
\mediaquery{max-textwidth=4cm}
{\ResponsiveSetup{characters=45}}
{\ResponsiveSetup{characters=60}}
\end{verbatim}

A media query can be declared using the \cs{mediaquery} command, which
expects three parameters -- the first is a list of tests, the next parameter
expects the code to be executed if the tests evaluate to true, and the last one
contains the code to be executed if the condition is not met. The code can
include the \cs{ResponsiveSetup} command, as well as any other commands. For
example, setting the size of the text block, header, and footer using the
\tbcode{Geometry} package.

\begin{figure*}[htpb]

  \centering

\begin{subfigure}[b]{0.45\textwidth}
    \centering
\fbox{%
\begin{minipage}{5cm}
\mediaquery{max-textwidth=4cm}
{\ResponsiveSetup{characters=45}}
{\ResponsiveSetup{characters=60}}
\setsizes{}

\smalllipsum

\end{minipage}}
\caption{Text width 5 cm}
\end{subfigure}
\hfill
  \begin{subfigure}[b]{0.45\textwidth}
    \centering
\fbox{%
\begin{minipage}{3.9cm}
\mediaquery{max-textwidth=4cm}
{\ResponsiveSetup{characters=45}}
{\ResponsiveSetup{characters=60}}
\setsizes{}

\smalllipsum

\end{minipage}}
\caption{Text width 3.9 cm}
\end{subfigure}
  \caption{Media Query Example}\label{fig:mediaguery}
\end{figure*}

We can test the following page properties: \texttt{paperwidth} and
\texttt{paperheight} for page dimensions, \texttt{textwidth} and
\texttt{textheight} for text dimensions, \texttt{orientation} for text
orientation, and \texttt{twocolumn} for detecting the use of two-column text in
the document.

Tests for text and page dimensions also support the prefixes \texttt{max-} and
\texttt{min-}. Using these, we can test whether a given dimension is smaller or
larger than a specified value.

For example, the following command changes the text color to blue if the
document has landscape orientation, the text width is less than 20 cm, and two
columns are used.

\begin{verbatim}
\mediaquery{orientation=landscape,
max-textwidth=20cm,
twocolumn=true}{\color{blue}}{}
\end{verbatim}




\subsection{The \tbcode{Linebreaker} Package}


\newcommand\testbox[1]{%
  \parbox{120pt}{%
    \parindent=15pt%
    \tolerance=1%
    \pretolerance=1%
    #1
  }%
}

\newcommand\printtest[1]{%
  \linebreakerdisable%
  \begin{subfigure}[b]{.45\textwidth}
    \centering
  \noindent\testbox{%
    #1
  }%
  \caption{Without the \tbcode{Linebreaker} package}
  \end{subfigure}
  \linebreakerenable%
  \hfill%
  \begin{subfigure}[b]{.45\textwidth}
    \centering
  \testbox{%
    #1
  }%
  \medskip
  \caption{With the \tbcode{Linebreaker} package}
  \end{subfigure}
}

The \tbcode{Linebreaker} package~\cite{linebreaker} prevents text from
overflowing in boxes and paragraphs. An example of its usage can be seen in
Figure~\ref{fig:linebreaker}, where it prevents several lines from overflowing
when typeset into a narrow column.


\tbcode{Linebreaker} utilizes  Lua\TeX's callback 
which controls line breaking. It replaces the default line
breaking function with a modified version that detects overflow or underflow in
the broken text. Upon detecting this problem, it retypesets the text with
increased values of \cs{tolerance} and \cs{emergencystretch} until the
overflow is suppressed or the maximum \cs{tolerance} limit is reached.

The advantage is that these changes to \cs{tolerance} and
\cs{emergencystretch} are local only to the currently broken paragraph and
do not affect the rest of the text.


\subsubsection{\tbcode{Linebreaker} Configuration}

The \tbcode{Linebreaker} package can be configured by specifying package
options using 
\cs{usepackage}\allowbreak\texttt{[<options>]\allowbreak\{linebreaker\}} or later in the
document body with the command 
\cs{linebreakersetup}\allowbreak\texttt{\{<options>\}}:

\begin{description}
  \item[maxcycles] – the number of attempts to re-typeset a paragraph.
  \item[maxemergencystretch] – the maximum value of \cs{emergencystretch}.
  \item[maxtolerance] – the maximum value of \cs{tolerance}.
\end{description}

\begin{verbatim}
\linebreakersetup{
% default 8189
maxtolerance = 90, 
% default 3em
maxemergencystretch = 1em, 
% default 30
maxcycles = 4              
}
\end{verbatim}



\begin{figure*}[htbp]
  \printtest{
The example document given below creates two pages by using Lua code alone. You
will learn how to access TeX's boxes and counters from the Lua side, shipout a
page into the PDF file, create horizontal and vertical boxes (hbox and vbox),
create new nodes and manipulate the nodes links structure. 
  }
  \caption{Example of using the \tbcode{Linebreaker} package}
  \label{fig:linebreaker}
\end{figure*}

\subsection{Other Packages Useful for the Automatic Typesetting}

We have demonstrated the use of the \tbcode{Responsive} and
\tbcode{Linebreaker} packages for automatic typesetting. These are not the only
useful packages that leverage the power of LuaTeX for automatic typesetting.
Noteworthy examples include \tbcode{Lua-widow-control} for suppressing widows
and orphans, and \tbcode{Luavlna}, which addresses certain typographical issues
for Czech and Slovak, while also preventing line breaks in SI units or academic
titles.

\section{Summary}


I hope the demonstration of the \tbcode{Rmodepdf} program caught your interest.
Even if it didn't, I believe that the side products developed alongside it are
useful on their own. 

This includes the capability to process HTML files using the \tbcode{LuaXML}
package and convert them to \LaTeX\ using its \tbcode{luaxml-transform}
library. The \tbcode{Responsive} package allows you to declaratively set the
document design depending on the currently chosen page size. Lastly, the
\tbcode{Linebreaker} package prevents line overflow, which is a common issue
especially in documents with fewer characters per line.



% \printbibliography
\bibliographystyle{tugboat}
% \nocite{book-minimal}      % make the example bibliography non-empty
\bibliography{responsive}       % xampl.bib comes with BibTeX
\makesignature

% \begin{summary}
%   This article focuses on the use of responsive design techniques to display
%   web pages on devices with different display sizes, such as mobile phones,
%   tablets, large monitors and printers. These methods allow optimizing the
%   readability of a document on all devices by using different font sizes,
%   individual page elements, and margins.

%   We present how similar functionality can be achieved using \LaTeX.
%   Specifically, it focuses on the use of Lua\LaTeX{} for automated typesetting,
%   using packages \tbcode{Responsive} for setting font size and line spacing according to page size,
%   \tbcode{Luavlna} to prevent the occurrence of
%   single-letter prepositions at line breaks, \tbcode{Lua-widow-control} to
%   reduce orphan lines at page breaks and page starts, and \tbcode{Linebreaker}
%   to prevent line overflow.

%   With these methods, a single source document can be used for different
%   outputs, such as print versions, e-book readers, and web pages, and achieve
%   optimal document display on all devices.

% \keywords: automatic typesetting, responsive design, Lua\LaTeX
% \end{summary}
\end{document}
