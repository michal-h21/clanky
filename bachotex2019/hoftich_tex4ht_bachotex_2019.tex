\documentclass{ltugproc}

\usepackage[hybrid]{markdown}
\newcommand\term[1]{\textit{#1}}
\newcommand\command[1]{\texttt{#1}}
\newcommand\packagename[1]{\texttt{#1.sty}}
\newcommand\option[1]{\texttt{-\/-#1}}
% \usepackage{microtype}
\newcommand\texfourht{\term{\TeX4ht}}
\newcommand\texfourhtcmd{\command{tex4ht}}
\newcommand\tfourhtcmd{\command{t4ht}}
\newcommand\makefourht{\command{make4ht}}
% \newcommand\DVI{\acro{DVI}}
\newcommand\extension[1]{\texttt{.#1}}
% make4ht extensions
\newcommand\mkextension[1]{\texttt{#1}}
\newcommand\switch[1]{\texttt{-\/-#1}}

\newcommand\nazev[1]{#1}
\newcommand\parencite[1]{#1}
\newcommand\prikaz[1]{\texttt{#1}}

\author{Michal Hoftich}
\title{\texfourht: LaTeX to Web publishing}
\address{Charles University, Faculty of Education}
\netaddress{michal.hoftich@pedf.cuni.cz}
\personalURL{https://www.kodymirus.cz}
\begin{document}

\begin{abstract}
  TODO: add abstract
\end{abstract}
\maketitle

\section{Overview of the conversion process}
\texfourht\ is a system for conversion of \TeX\ documents to various output
formats. Most notably \HTML, or \term{OpenDocument Format}, supported by word processors such as Microsof Word or LibreOffice
Writer. Overview of the system is depicted in the figure \ref{fig:overview}.


The package \packagename{tex4ht} starts the conversion process. The document
preamble is loaded as usual, but it keeps track of all loaded files. 
It loads special configuration files for the
used packages supported by \texfourht after the begin of the document.
These configuration with are named as the configured file with a
\extension{4ht} extension.  They can fix clashes between the configured package
and \texfourht, but most notably the package commands can be patched to insert
special marks to the \DVI\ file, so-called
hooks. 

After the package configuration, another type of \extension{4ht} files are loaded.
They populate inserted hooks with tags in the selected output format. In the
last step before processing of the document contents, a \extension{cfg}
provided by the user can configure the hooks with custom tags. Compilation of
the document then continue as usual, resulting in a special \DVI\ file.

The generated \DVI\ file is then processed with the \texfourhtcmd\ command.
This command creates output files, converts various input encoding to UTF-8,
and creates two auxiliary files: \extension{.idv} file is a special \DVI\ file that contains
pages to be converted to images. These can be the contents of the \LaTeX\ picture
environment or complex mathematics. The other file with the \extension{lg}
file contains list of output files, CSS instructions, and instructions for
compiling individual pages in the  \extension{idv} file to images.

The last step in the compilation chain is the \tfourhtcmd.
It process the \extension{lg} file and extracts the \acro{CSS} instructions,
converts the images in the \extension{idv} file and can call various external
commands.


\begin{figure*}[hbt!]
  \includegraphics[width=\textwidth]{img/tex4ht_process.pdf}
\caption{\texfourht\ process overview}
\label{fig:overview}
\end{figure*}

\section{Supporting scripts}

Because the entire conversion process consists of several consecutive steps,
we use scripts to make this process easier. The \texfourht\ distribution
contains several of such scripts. They differ in the supported output format, \TeX\ engine used 
or options passed to the underlying commands.
The most commonly used script  is \command{htlatex}, which uses the \PDF\TeX\ engine with
\LaTeX\ and produces the \HTML\ output. 

Each of the scripts loads the \texfourht\ package without need to specify it in
the document and options from the command line are passed to the package as well to
\texfourhtcmd\ and \tfourhtcmd commands.

For example the following command can be used to request the output in the
\acro{XHTML} format in the \acro{UTF-8} encoding:

\begin{verbatim}
htlatex filename.tex \
"xhtml,charset=utf-8"\
"-utf8 -cunihtf"
\end{verbatim}

However, these scripts are inflexible, each time they execute a three-time
compilation of the \TeX\  document. This will ensure the correct structure of
hypertext links and tables, as they require multiple compilations to
function properly. In the following steps, the document is processed by
\texfourhtcmd\ and \tfourhtcmd. 

For example, if the document contains a bibliography or
glossary that are created by external programs, it is necessary to first call
\command{htlatex}, then the desired program, and then \command{htlatex} again.
In the case of larger documents, compilation time in this way may be relatively
long.

The option passing to the underlying commands is also quite difficult.

New build scripts had been created for these reasons. First project that
attempted to simplify the \texfourht\ compilation process was
\command{tex4ebook}. It added support for e-books, in particular in
\term{ePub}, \term{ePub3} and \term{mobi} formats. It added support for use of
common command line switches and build files written in the \term{Lua} language.

The main difference between \command{tex4ebook} and \texfourht\ is the third compilation
step. The \tfourhtcmd\ command is used only to create a CSS file. 
Image conversion and execution of the external commands is controlled by the
\command{tex4ebook} itself. In addition, thanks to the build file support, it
is possible to execute commands between individual \TeX\ compilations. For
example, to execute an index processor or \command{bibtex}
after the first compilation.

The library that added the build support provided 
features useful also for other output formats than e-books. It
was extracted as a standalone tool and the \command{make4ht} build system is now a
recommended tool for the \texfourht\ use.


\section{\command{make4ht} build system}

\makefourht\ enables creation of build scripts in the Lua language. 
It supports the execution of arbitrary commands during the conversion,
post-processing of the generated files or setting command for the image
conversion.
Using using so-called modes, it is possible to 
influence the order of compilation using switches directly from the   command
line. For example, the basic script used by \makefourht\ supports the draft mode, which
only runs one compilation of the document instead of the usual three. This can
be used to significantly  speed up of the  compilation. 

Currently, only \LaTeX\ is supported, Plain\TeX\ support  is possible, but it is
more complicated and \ConTeXt\ is not supported at all. In the following we
will focus only on \LaTeX.

make4ht supports a number of switches and options that affect the progress of compiling and processing of the output files.
make4ht can be launched as follows:

\bgroup\small
\begin{verbatim}
make4ht [switches for make4ht] file.tex \
"options for tex4ht.sty" "switches for tex4ht"\
"switches for t4ht" "switches for TeX" 
\end{verbatim}
\egroup


This complicated list comes from  the way \command{htlatex} is  functioning, 
as it needs passing options for all components involved in
compilation. In most cases, fortunately, there is no need to use all options.
Most of the properties that \texfourhtcmd\ and \tfourhtcmd\ provide can be requested using the
\makefourht\ switches.

\subsection{\makefourht\ command line switches}

All command line switches \makefourht\ supports have a short and long version. In
addition, short switches can be combined together. The following two commands
are identical:


\bgroup\small
\begin{verbatim}
make4ht --lua --utf8 --mode draft filename.tex
make4ht -lum draft filename.tex
\end{verbatim}
\egroup

This command uses Lua\LaTeX\ for the compilation, which will be executed only one
once thanks to the draft mode. The resulting document will be in \acro{UTF-8}
text encoding. The default output format used by \makefourht\ is \HTML5. \command{htlatex} on the other hand uses \HTML4.

In addition to the \switch{lua} , \switch{utf8}, and \switch{mode} switches, there are a number of other useful switches:


\begin{description}
  \item[\switch{config (-c)}] configuration file for \texfourht. Allows user to change tags inserted into output files.
  \item[\switch{build-file (-e)}] select a build file.
  \item[\switch{output-dir (-d)}] the directory where the output files will be copied.
  \item[\switch{shell-escape (-s)}] will use the \texttt{-shell-\/escape} option for \LaTeX, enabling execution of the external commands.
  \item[\switch{xetex (-x)}] the document will be compiled using Xe\LaTeX.
  \item[\switch{format (-f)}] select an output format.
\end{description}

There are other switches, but these are most useful for a common use.


\subsection{Output formats and extensions}

\texfourht\ supports a wide range of \XML-based formats, from \acro{XHTML}, through \acro{ODT} to
\acro{DocBook} and \acro{TEI}. 

The \switch{format} switch supports the following formats: \command{html5}, \command{xhtml}, \command{odt}, \command{tei} and \command{docbook} (it is necessary to use the lowercase letters). 
The default format is \command{html5}. Formats can be selected also using the \packagename{tex4ht} option:

\begin{verbatim}
make4ht filename.tex "docbook"
\end{verbatim}

The positive feature of the \switch{format} is that it can fix some common issues for the particular formats. 
It can also load extensions. Extensions allow us to influence the compilation
without having to use a build script. The list of extensions to be used can be
written after the format name. They can be  enabled using the plus character,
disabled with the minus character disabled\footnote{Extensions can also be
enabled from the \makefourht\ configuration file, it can be useful to be able
to disable them from  the command line.}. The following command uses the \HTML
Tidy command to fix some common errors in the generated \HTML\ file:


\begin{verbatim}
make4ht -f html5+tidy simple-example.tex
\end{verbatim}

The following extensions are available:

 

\begin{description}
  \item[latexmk\_build] uses the \command{latexmk} command to build a document. It will take care of the  calling of external commands. For example, it can be used to create a bibliography.

  \item[tidy] cleans the HTML file with the tidy command.

  \item[dvisvgm\_hashes] efficient generation of images using the dvisvgm command. It can use multiple processor cores and only creates images that have been changed or created since the last compilation. This makes the compilation noticeably faster.  

  \item[common\_filters and common\_domfilters]
clean the document using filters. The filters will be discussed later in the  text.

\item[mathjaxnode] convert \MathML\ math code into special \HTML\ using
  \term{MathJax Node Page}\footnote{\url{https://github.com/pkra/mathjax-node-page/}}.
  It produces mathematics that can be viewed in web browsers without \MathML
    support. The rendering of the result doesn't need JavaScript, which results in much faster display of the document compared  to the regular MathJax.

\item[staticsite] creates a document that can be used for static page generators such as \term{Jekyll}\footnote{\url{https://jekyllrb.com/}}. These are useful for creating a blog or similar more complex website.
\end{description}
    

Extensions can be further configured. This brings us to the configuration file and build scripts. 



\subsection{Configuration file for make4ht}

\makefourht\ supports build scripts in the  Lua language. They can be used to
call external commands, pass parameters to executed command, apply filters to
the output files, affect image conversion, or configure extensions.

The \extension{make4ht} configuration file is a special build script that is loaded
automatically and should contain only general configurations shared between documents. In contrast,
normal build files may contain configurations useful only for the current document. The
configuration file can be located in the  directory of the current   document or in  its
parent directories. 

This can be useful, for example, for maintaining a blog.
Each document is located in its own directory. In the parent directory
can be located a configuration file to ensure proper processing. Here's a small
example: 

\begin{verbatim}
filter_settings "staticsite" {
  site_root = "output" 
}

Make:enable_extension("common_domfilters")
if mode=="publish" then
  Make:enable_extension("staticsite")
  Make:htlatex {}
end

\end{verbatim}

This configuration file sets the option \verb|site_root| for the \mkextension{staticsite} extension using the \verb|filter_settings| command. 
This command can be used to set options for filters but also for extensions.
The name of the filter or extension is separated from the command by a space,
followed by another space-separated field, where options can be set.

Another used command is \verb|Make:enable_extension|, which enables the
extension. In this case the \verb|common_domfilters| extension is used in every
compilation, but the \mkextension{staticsite} extension is used only in the  \term{publish}
mode. In this mode it is also necessary to use the \verb|Make:htlatex{}| to
require at least one \LaTeX\ compilation.

Now it is possible to run \makefourht\ in the publish mode:


\begin{verbatim}
make4ht -um publish simple-example.tex
\end{verbatim}

The \texttt{output} directory will be created if it does not already exist. \HTML\ and \CSS\ files will be copied here. The static site generator must be configured to look up for files here and it needs to be executed manually, the extension doesn't do that. 

The resulting HTML file may take the following form: 


\begin{verbatim}
---
time: 1544811015
date: '2018-12-14 18:10:47'
title: 'sample'
styles:
- '2018-12-14-simple-example.css'
meta:
- charset: 'utf-8'
---
<p>Sample document</p>
\end{verbatim}

The document header enclosed between the two \verb|---| contains variables in the \acro{YAML}
format extracted from   \HTML file. Only the contents of the document body
remains in the document, old header is stripped off. The
static generator can then create a page based on the template and the variables
in the \acro{YAML}  header.

This was just a basic example, filters and extensions have much wider configurable options, all of which are described in the \makefourht\  documentation\footnote{\url{https://ctan.org/pkg/make4ht}}.


\subsection{Build files}

In the compilation scripts it is possible to use the same procedures as in   the configuration
file, but more focused on the particular compiled document. 
The following code shows use of the DOM filters. These take advantage of the
\term{LuaXML}\footnote{\url{https://ctan.org/pkg/luaxml}} library. It allows to process XML files
using the Document Object Model (DOM) interface . This makes it easy to
navigate, edit, create or delete elements.

The use of DOM filters is shown on the following example for Lua\LaTeX:


\begin{verbatim}
\documentclass{article}
\begin{document}
Test {\itshape háčků}
\end{document}
\end{verbatim}


Because of the known error in processing the \DVI\ file with the \texfourhtcmd\ command, 
each accented character 
in the generated \HTML\ file 
will be placed in  a separate \verb|<span>| element:


\begin{verbatim}
<!--l. 4--><p class="noindent" >Test 
<span class="rm-lmri-10">h</span><span 
class="rm-lmri-10">á</span><span 
class="rm-lmri-10">čk</span><span 
class="rm-lmri-10">ů</span> </p> 
\end{verbatim}

The following build file removes this by using the built-in \mkextension{joincharacters} DOM
filter. In addition, it changes the value of class attribute for all \verb|<p>| elements to
\term{mypar}, just to show how to work with the  DOM Interface:

\begin{verbatim}
local domfilter = 
  require("make4ht-domfilter")

local function domsample(dom)
  -- the following code will process
  -- all <p> elements
  for _, par in 
    ipairs(dom:query_selector("p")) do
    -- set the "class" attribute
    par:set_attribute("class", "mypar")
  end
  return dom
end

local process = domfilter({
  "joincharacters", 
  domsample
})
Make:match("html$", process)
\end{verbatim}

The script uses the standard Lua \texttt{require} function to  load the
\texttt{make4ht-domfilter} library. This creates a \texttt{domfilter} function that takes a list
of DOM filters to execute as a parameter. 

Each call to the \texttt{domfilter} function
creates another function with a chain of filters specified in a table. 
Parameters in the   fields can be either the name of an existing DOM
filter, or a function defined in the file. 

The filter chain can be then used in the \texttt{Make:match}
function. It takes a filename pattern to match files for which the filters
should be executed, and the filter chain.


The \texttt{process} function will run
on each file whose filename ends with \texttt{html} in this case.

The resulting \HTML\ file does not contain the extra \verb|<span>| elements and the \verb|<p>| element has a class attribute value of \texttt{mypar}:

\begin{verbatim}
<!-- l. 3 --><p class='mypar'>
Test <span class='rm-lmri-10'>háčků</span> 
</p> 
\end{verbatim}

More complex build file with external commands execution or configuration of the image generation may look as follows:

\begin{verbatim}
Make:add("biber", "biber ${input}")
Make:htlatex {}
Make:biber {}
Make:htlatex {}
Make:image("png$")
"dvipng -bg Transparent -T tight "..
-o ${output} "-pp ${page} ${source}")
Make:match("html$")
"tidy -m -utf8 -asxhtml -q -i ${filename}")
\end{verbatim}


The \verb|make:add| command may be used to add a new command, like
\texttt{biber} in this case. The second parameter is a formatting string, which
may contain \verb|${variable name}| templates, which are replaced by parameters
set by \makefourht. The \verb|${input}| will be replaced with the input file name for example.

The added command can be then used using the \verb|Make:command name {}|
command. Additional variables may be set in the table passed as the argument.

The \texttt{Make:htlatex} command is built-in and requires one execution of \LaTeX\ with active \texfourht.

The \texttt{Make:image} command configures the image conversion. Three
variables are available, \texttt{page} contains the page number of the image in
the \DVI\ file, \texttt{output} is the name of the output image, and \texttt{source} is
the name of the \extension{idv} file.

The use of the \texttt{Make:match} command had been shown in the previous
example, but it may also contain string with the command to be executed. The
\texttt{filename} variable contains filename of the currently processed
generated file.




\section{\texfourht\ configuration}


Output format tags embedded in a document are fully configurable using several
mechanisms. The easiest way to use the \packagename{tex4ht} package options, more advanced choice is
to use a custom configuration file and the most powerful  option is to use the  \extension{4ht} files.

When a \TeX\ file is compiled using \makefourht\ or another compilation script, the
\packagename{tex4ht} package is loaded before the document itself. Package
options are obtained from the  compilation script arguments. As a result, it is not necessary
to insert the \packagename{tex4ht} package in the  document itself.

The file loading mechanism is modified to register each loaded file with
\texfourht. For some packages, \texfourht\ also contains code that prevents it from
loading, or to immediately override some macros. This is necessary for packages
that are incompatible with   \texfourht, such as \texttt{Fontspec}.

After execution of  the document preamble, the configuration files for the packages
detected during processing are loaded. These files are named as a filename of the configured package
without the extension plus the  \extension{4ht} extension. Their main function is to insert configurable
macros, called hooks, into the commands provided by the package. In general, it
is better not to redefine macros, only to patch them with the commands
TeX4ht provides for this purpose is enough in most cases.

Output format configuration files are loaded after the package configuration
files are processed. These define the contents of configuration macros, or hooks as we call them.
Not only the output format tags are inserted here, they can contain
any valid \TeX\ commands. 

\subsection{\packagename{tex4ht} options}

Many configurations are conditional, they are executed only in the presence of some options 
for \packagename{tex4ht} package.
Each output format configuration file can test any option, which means that there is no final list of possible options and 
each output format can support different set of options.

Because it is not recommended to put the \packagename{tex4ht} package directly into the document, it is possible to pass the options in other ways. 
The easiest way is to use the compilation script argument. It is always the argument following the document name:

\begin{verbatim}
make4ht filename.tex "mathml,mathjax"
\end{verbatim}

Options used in this case are \texttt{mathml} and \texttt{mathjax}. 

Another option is to pass the \cs{Preamble}  command  the private configuration file, which we'll show in the  next section.

As already mentioned, there is no final list of options, each output format can support any number of them.

However, we can show some choices regarding mathematical outputs in  \HTML.
Normal configuration for mathematical environments produces a blend of rich
text and images for more complex outputs, if that cannot be easily created with
HTML elements. Often this output doesn't look good. As an alternative,
it is possible to use images for all math content. This can be achieved by using the \texttt{pic-m}
options for inline mathematics and \texttt{pic-\meta{environment name}} for mathematical
environments. For example, the  \texttt{pic-align} option will require pictures for all \texttt{align} environments.

By default, images are created in the \acro{PNG} format. Higher quality can be
achieved using the \acro{SVG} vector format. This can be enforced using the
\texttt{svg} option.

\texfourht\ documentation is quite spartan. More comprehensive information
about the available configurations can be found in the  \extension{log} file after the compilation 
of a document using the \texttt{info} option.

Options listed  in the example above, \texttt{mathml} and \texttt{mathjax},
provide the best mathematical content output in. The \MathML\ markup language, requested by the first option,
encodes the mathematical information, but its support in Web browsers
is poor. The second option requests the \term{MathJax} library, which can render the \MathML\ output in all
browsers with the  JavaScript support.

\texttt{mathjax} option used without \texttt{mathml} completely turns off
compilation of math, all math content remains in the \HTML\ document as raw
\LaTeX\ macros. \term{MathJax} then processes the document and renders the math in the
correct way. The disadvantage of this method is that MathJax does not support
all packages and user commands, it needs a special configuration in these
cases. Sometimes it is not even possible to emulate more complex macros.




\subsection{Soukromý konfigurační soubor}

Pomocí soukromého konfiguračního souboru je možné vkládat vlastní obsah do
konfiguračních háčků. Tento soubor má zvláštní strukturu:

\begin{verbatim}
...Definice v~preamble...
\Preamble{volby pro tex4ht.sty}
... Normální konfigurace ...
\begin{document}
... Konfigurace pro hlavičku HTML souboru
\EndPreamble
\end{verbatim}

Příkazy \verb|\Preamble|, \verb|\begin{document}| a \verb|\EndPreamble| musí
být v~konfiguračním souboru obsaženy. Cesta k souboru může být předána
\nazev{make4ht} pomocí parametru \verb|-c|:

\begin{verbatim}
make4ht -c myconfig.cfg filename.tex
\end{verbatim}

Pokud konfigurační soubor není uložený v~aktuálním pracovním adresáři, je nutno
použít plnou cestu k souboru.


Existuje několik konfiguračních příkazů. Nejdůležitějším je \verb|\Configure|,
který nastavuje běžné konfigurace, dalšími jsou například \verb|\ConfigureEnv|
pro nastavení prostředí, nebo \verb|\ConfigureList| pro seznamy.

Příkaz \verb|\HCode| se používá pro vkládání značek výstupního formátu.
Součástí jeho argumentu může být příkaz \verb|\Hnewline| pro vložení zalomení
řádku. Instrukce \verb|\Css| vloží kód pro kaskádové styly do CSS souboru.


Následující ukázka použije element \verb|<em>| pro příkaz \verb|\textit|:

\begin{verbatim}
\Configure{textit}{\HCode{<em>}\NoFonts}{\EndNoFonts\HCode{</em>}}
\end{verbatim}

Příkaz \verb|\Configure| podporuje variabilní počet argumentů, záleží na
definici háčků, kolik argumentů je potřeba. Prvním argumentem je vždy název
konfigurace, další argumenty poté vkládají kód do háčků. V~typickém případě
vyžaduje konfigurace dva háčky -- jeden umístí kód před začátek příkazu, druhý
za jeho konec. Tak je tomu v~případě konfigurace pro \texttt{textit}. Název
konfigurace se může shodovat s~názvem konfigurovaného příkazu, jako v~tomto
případě, ale vždy záleží na konfiguračním \nazev{4ht} souboru pro daný balíček,
jaký název použije.


Příkaz \verb|\NoFonts| zakáže vkládání formátovacích elementů při zpracování
DVI souboru. \prikaz{tex4ht} vkládá automaticky tyto elementy při změně písma.
Díky tomu je možné vytvořit základní formátování i pro nepodporované příkazy
bez konfigurací, ovšem pro konfigurované příkazy je toto zbytečné. 



Komplikovanou problematikou jsou odstavce. \nazev{TeX4ht} je někdy vkládá na
nevhodná místa. Týká se to především konfigurací prostředí, která mohou
obsahovat několik odstavců a celý svůj obsah umísťují do jednoho elementu. Může
se stát, že počáteční značka odstavce je umístěna před začátkem tohoto elementu,
správně by ovšem měla následovat až po něm. Pro tento případ existují příkazy
\verb|\IgnorePar|, který zabrání vložení značky pro následující odstavec, a
\verb|\EndP|, který vloží zavírací značku pro předešlý odstavec. Existuje více
příkazů pro práci s~odstavci, ale tyto jsou nejdůležitější.

V~následujícím příkladu použijeme hypotetické prostředí \texttt{rightaligned},
které by mělo být umístěno v~elementu \verb|<article>|. 

\begin{verbatim}
\ConfigureEnv{rightaligned}
{\HCode{<section class="right">}}
{\HCode{</section>}}{}{}
\end{verbatim}

Příkaz \verb|\ConfigureEnv| očekává pět parametrů, prvním je název
konfigurovaného prostředí, druhým je kód vložený na začátku prostředí a třetím
kód vložený na jeho konci. Zbylé dva argumenty se používají pouze pokud je
konfigurované prostředí založeno na seznamu, ve většině případů mohou zůstat prázdné. 
HTML kód vytvořený touto konfigurací může vypadat následovně:

\begin{verbatim}
<p class="indent" >   <section class="right">
...
</p><p class="indent></section>
\end{verbatim}

Tento kód je nevalidní, protože ukončovací značka pro element \verb|<p>| je
umístěna na špatné úrovni zanoření. Nevalidní kód může způsobit přerušení běhu
DOM filtrů a jiných nástrojů, které zpracovávají výsledné XML soubory. Této
situaci je třeba předcházet.

Správná konfigurace je poněkud komplikovanější:


\begin{verbatim}
\ConfigureEnv{rightaligned}
{\ifvmode\IgnorePar\fi\EndP\HCode{<section class="right">}\par}
{\ifvmode\IgnorePar\fi\EndP\HCode{</section>}}{}{}
\end{verbatim}

V~tomto případě řídíme vkládání značek pro odstavce sami a výsledek je v~pořádku:


\begin{verbatim}
<section class="right">
<!--l. 9--><p class="indent" >
...
</p></section>
\end{verbatim}

V~konfiguracích můžeme také vyžádat konverzi části dokumentu na obrázek. Toho
se dá docílit pomocí příkazů \verb|\Picture*|, respektive \verb|\Picture+|.
Rozdíl mezi nimi je ten, že první zpracovává svůj obsah jako vertikální box.
V~každém případě se obsah mezi těmito příkazy a ukončovacím příkazem
\verb|\EndPicture| převede na obrázek.

Těchto příkazů lze využít například pro konverzi složitější matematiky nebo diagramů,
ale čistě technicky podporují veškerý obsah, který dokáže zpracovat použitý
konvertor z~DVI do obrazových souborů, typicky \nazev{Dvipng} nebo
\nazev{Dvisvgm}.

Následující příklad vytvoří obrázek pro text obsažený v prostředí \nazev{topicture}:

\begin{verbatim}
\documentclass{article}
\newenvironment{topicture}{\bfseries}{}
\begin{document}
\begin{topicture}
Obsah tohoto prostředí by měl být zobrazen jako obrázek
\end{topicture}
\end{document}
\end{verbatim}

Konfigurace pro prostředí \nazev{topicture} využije příkazu \verb|\Picture*|:

\begin{verbatim}
\ConfigureEnv{topicture}{\Picture*{}}{\EndPicture}{}{}
\end{verbatim}


\section{Závěr}

Možnosti konfigurace \nazev{TeX4ht} jsou rozsáhlé, v~předešlém textu jsme se
dotkli pouze základů, které by však měly vystačit pro řešení nejběžnějších
otázek, které uživatelé tohoto systému řeší. Vynechali jsme také ukázky, jakým
způsobem lze přidat konfiguraci pro nový \LaTeX ový balíček. Tato témata by
vydala na rozsáhlé samostatné články.

Díky sestavovacímu systému \prikaz{make4ht} je použití celého systému snazší a
efektivnější než tomu bylo v~minulosti.

V~současné době probíhá za finanční podpory poskytnuté \nazev{Českoskovenským
sdružením uživatelů \TeX u} tvorba nové dokumentace pro \nazev{TeX4ht}, kde
budou témata tohoto článku probrána do větší hloubky, spolu s~konkrétními
příklady užití. 


% \printbibliography
% \begin{summary}

%   The article gives overview of the current state of development of TeX4ht,
%   \LaTeX\ to XML convertor. It introduces \prikaz{make4ht}, a build system for
%   TeX4ht as well as basic ways how to configure TeX4ht.

% \end{summary}


\end{document}
