\documentclass{beamer}

\usepackage[czech]{babel}

\title{LaTeX na web: Tvorba HTML pro knihy, blogy a prezentace s TeX4ht}


%    "LaTeX na web: Tvorba HTML pro knihy, blogy a prezentace s TeX4ht"

%    "Převod LaTeXu na HTML: Praktické šablony pro weby, blogy a prezentace"

%    "Od LaTeXu k HTML: Jak generovat webové stránky, blogy a handouty s TeX4ht"

%    "Efektivní konverze LaTeXu do HTML: Tvorba webů, blogů a prezentací s TeX4ht"

%    "TeX4ht v praxi: Jak převést LaTeX do HTML pro knihy, blogy a prezentace"

\author{Michal Hoftich}
\date{}

\begin{document}

\frame{\titlepage}

\begin{frame}
\frametitle{Abstrakt}
V přednášce se podíváme na praktickou práci se třemi šablonami pro nástroj
TeX4ht, který slouží k převodu LaTeXu na HTML. Ukážeme si, jak pomocí
jednoduchých šablon generovat webové stránky, blogové příspěvky a HTML verze
rezentací. Představíme konkrétní příklady konverze LaTeX dokumentů pro různé
účely: tvorbu webu pro knihu, publikování článků na blogu a generování handoutů
z Beamer prezentací. Všechny ukázky budou zaměřeny na jednoduchost a
efektivitu, takže i ti, kteří nemají hlubší znalosti HTML nebo CSS, si budou
moci snadno přizpůsobit LaTeX pro své potřeby. Naučíte se, jak si přizpůsobit
šablony, aby odpovídaly konkrétním požadavkům a jak jednoduše převádět různé
typy dokumentů do formátu HTML bez nutnosti ručního zásahu do kódu. Přednáška
je určena pro všechny, kteří chtějí efektivně pracovat s LaTeXem a získat
praktické dovednosti pro publikování na webu.

Prakticky se zaměříme na tři šablony:
\begin{itemize}
    \item Webové stránky pro knihy
    \item Blogové příspěvky
    \item Handouty z Beamer prezentací
\end{itemize}
Přednáška je určena i pro ty, kteří nemají hlubší znalosti HTML nebo CSS.
\end{frame}

\begin{frame}
\frametitle{Struktura přednášky}
\begin{enumerate}
    \item Úvod
    \item TeX4ht: základní představení
    \item Tři šablony (kniha, blog, handout)
    \item Závěr a diskuse
\end{enumerate}
\end{frame}

\section{Úvod}
\begin{frame}
\frametitle{Úvod (5 minut)}
\begin{itemize}
    \item Co je LaTeX a proč je oblíbený pro tvorbu dokumentů.
    \item TeX4ht: nástroj pro převod LaTeXu do HTML.
    \item Význam exportu dokumentů do různých formátů (web, blog, prezentace).
\end{itemize}
\end{frame}

\section{TeX4ht: základní představení}
\begin{frame}
\frametitle{Představení TeX4ht (5 minut)}
\begin{itemize}
    \item Co je TeX4ht a jak funguje.
    \item Výhody oproti jiným nástrojům.
    \item Význam šablon při převodu dokumentů.
\end{itemize}
\end{frame}

\section{Šablony}

\subsection{Web pro knihu}
\begin{frame}
\frametitle{První šablona: Web pro knihu (8 minut)}
\begin{itemize}
    \item Cíl: Strukturovaný web, každá kapitola samostatná HTML stránka.
    \item Ukázka základního LaTeX souboru pro knihu.
    \item Použití šablony TeX4ht pro převod.
    \item Výsledná podoba webu.
\end{itemize}
\end{frame}

\begin{frame}[fragile]
\begin{itemize}
  \item obsah na hlavní stránce má navíc třídu \verb|bookTOC|, aby se umožnilo konfigurovat ho zvlášť
  \item konfigurovat seznam sekcí pro obsah pomocí 
\end{itemize}
\begin{verbatim}
\Configure{fulltocsections}{chapter,likechapter,section,likesection,subsection,likesubsection}
\end{verbatim}
\end{frame}

- 


\subsection{Blogové příspěvky}
\begin{frame}
\frametitle{Druhá šablona: Blogové příspěvky (8 minut)}
\begin{itemize}
    \item Cíl: Převod LaTeX dokumentů na blogové články.
    \item Ukázka LaTeX souboru pro blog.
    \item Konverze na HTML pomocí šablony.
    \item Výsledný HTML kód blogového příspěvku.
\end{itemize}
\end{frame}

\subsection{Handout z prezentace}
\begin{frame}
\frametitle{Třetí šablona: Handout z Beamer prezentace (5 minut)}
\begin{itemize}
    \item Cíl: Tvorba HTML handoutu z Beamer prezentace.
    \item Ukázka jednoduché Beamer prezentace.
    \item Použití šablony TeX4ht pro převod.
    \item Výsledný HTML soubor pro tisk.
\end{itemize}
\end{frame}

\section{Závěr}

\begin{frame}
\frametitle{Závěr a Q\&A (4 minuty)}
\begin{itemize}
    \item Shrnutí ukázaných postupů.
    \item Diskuze o dalších možnostech využití TeX4ht.
    \item Prostor pro dotazy účastníků.
\end{itemize}
\end{frame}

\end{document}
