\documentclass{beamer}

\usepackage[czech]{babel}

\title{LaTeX na web: Tvorba HTML pro knihy, blogy a prezentace s TeX4ht}
% \title{Publikace LaTeXových dokumentů na web pomocí TeX4ht a GitHub Actions}


%    "LaTeX na web: Tvorba HTML pro knihy, blogy a prezentace s TeX4ht"

%    "Převod LaTeXu na HTML: Praktické šablony pro weby, blogy a prezentace"

%    "Od LaTeXu k HTML: Jak generovat webové stránky, blogy a handouty s TeX4ht"

%    "Efektivní konverze LaTeXu do HTML: Tvorba webů, blogů a prezentací s TeX4ht"

%    "TeX4ht v praxi: Jak převést LaTeX do HTML pro knihy, blogy a prezentace"

\author{Michal Hoftich}
\date{}

\begin{document}

\frame{\titlepage}

\begin{frame}
\frametitle{Abstrakt}

Přednáška představí sadu šablon pro nástroj TeX4ht, který slouží k převodu
LaTeXových dokumentů do HTML. Tyto šablony výrazně usnadňují publikaci různých
typů dokumentů na webu a přinášejí moderní možnosti zpracování a automatizace.

První šablona je určena pro převod knižních dokumentů do webové podoby.
Umožňuje rozdělení textu do jednotlivých kapitol s automaticky generovanou
navigací a podporou responzivního designu, takže je výsledek dobře čitelný i na
mobilních zařízeních.

Druhá šablona slouží k tvorbě staticky generovaných blogů. Každý příspěvek je
psán jako samostatný LaTeXový dokument, který je pomocí TeX4ht převeden do
HTML. Následně jsou tyto články zpracovány statickým generátorem webů, jako je
například Jekyll, který se postará o sestavení celého blogu, vytvoření
rozcestníků, archivů a další navigace.

Třetí šablona je zaměřena na převod prezentací vytvořených v prostředí Beamer
do formy tzv. handoutů – přehledových materiálů pro posluchače. Výsledkem je
čitelný a dobře strukturovaný webový dokument vhodný pro sdílení po přednášce.

Všechny šablony jsou navrženy tak, aby fungovaly v rámci GitHub Actions. To
znamená, že dokumenty mohou být automaticky zkompilovány a publikovány online
pokaždé, když dojde ke změně v repozitáři. Tento přístup zajišťuje, že je
webová verze dokumentu vždy aktuální.




\end{frame}

\begin{frame}
\frametitle{Struktura přednášky}
\begin{enumerate}
    \item Úvod
    \item TeX4ht: základní představení
    \item Tři šablony (kniha, blog, handout)
    \item Závěr a diskuse
\end{enumerate}
\end{frame}

\section{Úvod}
\begin{frame}
\frametitle{Úvod (5 minut)}
\begin{itemize}
    \item Co je LaTeX a proč je oblíbený pro tvorbu dokumentů.
    \item TeX4ht: nástroj pro převod LaTeXu do HTML.
    \item Význam exportu dokumentů do různých formátů (web, blog, prezentace).
\end{itemize}
\end{frame}

\section{TeX4ht: základní představení}
\begin{frame}
\frametitle{Představení TeX4ht (5 minut)}
\begin{itemize}
    \item Co je TeX4ht a jak funguje.
    \item Výhody oproti jiným nástrojům.
    \item Význam šablon při převodu dokumentů.
\end{itemize}
\end{frame}

\section{Šablony}

\subsection{Web pro knihu}
\begin{frame}
\frametitle{První šablona: Web pro knihu (8 minut)}
\begin{itemize}
    \item Cíl: Strukturovaný web, každá kapitola samostatná HTML stránka.
    \item Ukázka základního LaTeX souboru pro knihu.
    \item Použití šablony TeX4ht pro převod.
    \item Výsledná podoba webu.
\end{itemize}
\end{frame}

\begin{frame}[fragile]
\begin{itemize}
  \item obsah na hlavní stránce má navíc třídu \verb|bookTOC|, aby se umožnilo konfigurovat ho zvlášť
  \item konfigurovat seznam sekcí pro obsah pomocí 
\end{itemize}
\begin{verbatim}
\Configure{fulltocsections}{chapter,likechapter,section,likesection,subsection,likesubsection}
\end{verbatim}
\end{frame}

- 

\begin{verbatim}
% konfigurace šířky menu a textu
% druhej argument je CSS jednotka, je třeba, aby se používala stejná jednotka, protože se s těma šířkama počítá v media queries
\Configure{bodywidth}{80}{ch}
\Configure{maintocwidth}{30}{ch}
\end{verbatim}


\subsection{Blogové příspěvky}
\begin{frame}
\frametitle{Druhá šablona: Blogové příspěvky (8 minut)}
\begin{itemize}
    \item Cíl: Převod LaTeX dokumentů na blogové články.
    \item Ukázka LaTeX souboru pro blog.
    \item Konverze na HTML pomocí šablony.
    \item Výsledný HTML kód blogového příspěvku.
\end{itemize}
\end{frame}

\subsection{Handout z prezentace}
\begin{frame}
\frametitle{Třetí šablona: Handout z Beamer prezentace (5 minut)}
\begin{itemize}
    \item Cíl: Tvorba HTML handoutu z Beamer prezentace.
    \item Ukázka jednoduché Beamer prezentace.
    \item Použití šablony TeX4ht pro převod.
    \item Výsledný HTML soubor pro tisk.
\end{itemize}
\end{frame}

\section{Závěr}

\begin{frame}
\frametitle{Závěr a Q\&A (4 minuty)}
\begin{itemize}
    \item Shrnutí ukázaných postupů.
    \item Diskuze o dalších možnostech využití TeX4ht.
    \item Prostor pro dotazy účastníků.
\end{itemize}
\end{frame}

\end{document}
