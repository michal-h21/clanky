\documentclass[czech]{beamer}
\usetheme{metropolis}
%\usepackage[T1]{fontenc}
%\usepackage[utf8]{inputenc}
\usepackage{babel}
\newtheorem{priklad}{Příklad}
\newtheorem{pozor}{Pozor!}
\addtobeamertemplate{block begin}{\setlength\abovedisplayskip{0pt}}{}
\addtobeamertemplate{block end}{\vspace*{-1em}}{}
\newcommand\prepinac[1]{\texttt{-\/-#1}}


\newcommand\myfig[3][width=.9\textwidth]{%
  \figure\includegraphics[#1]{#2}%
  \caption{#3}%
\endfigure}

\usepackage{graphicx}
\title{Publikování z~\LaTeX u na web}
\author{Michal Hoftich\inst{1}}
\institute{
  \inst{1}
  <\url{michal.h21@gmail.com}>\\
  Ústřední knihovna PedF UK
}

\date{Přednáška pro CSTUG 2018}

\begin{document}

\frame{\titlepage}

\begin{frame}
  \frametitle{Obsah}
  \tableofcontents
\end{frame}


\section{Konverzní systém tex4ht}


\begin{frame}
  \frametitle{Historie tex4ht}
  \begin{itemize}
    \item \url{https://www.tug.org/tex4ht/}
    \item původní autor Eitan Gurari (1947--2009)
    \item systém vzniká od poloviny 90. let
  \end{itemize}
\end{frame}

\begin{frame}
  \frametitle{Další konvertory do HTML}
  \begin{itemize}
    \item Pandoc
    \item LaTeXML
    \item LaTeX2HTML
    \item Lwarp
  \end{itemize}
\end{frame}

\begin{frame}

  \frametitle{Základní popis systému TeX4ht}
  \begin{itemize}
    \item systém se skládá z množství kompilačních skriptů, které se ale liší
      jen v přednastavených hodnotách
  \item základním skriptem byl \texttt{htlatex} 
  \item v současné době byl plně nahrazen \texttt{make4ht}
\end{itemize}
\end{frame}


\begin{frame}
  \frametitle{Tři fáze kompilace}
  \begin{itemize}
    \item Kompilace dokumentu TeXem s nahraným souborem tex4ht.sty
    \item Zpracování dvi souboru programem tex4ht
    \item Zpracování .lg souboru programem t4ht
\end{itemize}
\end{frame}

\begin{frame}
  \frametitle{Kompilace dokumentu TeXem s nahraným souborem tex4ht.sty}
  \begin{itemize}
    \item pro podporované balíčky jsou nahrané \texttt{.4ht} soubory, které vkládají
        konfigurovatelné háčky na vhodná místa
    \item po vložení háčků se nahrají jejich konfigurace v závislosti na
        výstupním formátu
  \end{itemize}
\end{frame}

\begin{frame}
  \frametitle{Zpracování dvi souboru programem tex4ht}
  \begin{itemize}
    \item   zápis výstupních souborů
    \item   konverze znakových sad
      \begin{itemize} 
        \item  poměrně komplikovaný proces, potřebujeme
          doplňkové soubory pro fonty obsahující unicode entity nebo ASCII
          znaky pro jednotlivé znaky písma
  \end{itemize}
    \item   zachovává základní styly písem, podporuje jakákoli makra měnící vzhled písma
    \item   příprava .lg souboru
    \item   zápis .idv souboru obsahující stránky pro konverzi na obrázky
  \end{itemize}
\end{frame}

\begin{frame}
  \frametitle{Zpracování .lg souboru programem t4ht}
  \begin{itemize}
    \item   výstup do CSS souboru
    \item   konverze vložených obrázků z .dvi do výstupního formátu
    \item   zpracování výstupních souborů externími programy 
    (\texttt{xslt}, \texttt{tidy}, \texttt{xmllint}, \texttt{xtpipes})
    \item   kopírování souborů do cílového adresáře
  \end{itemize}
\end{frame}

\begin{frame}[fragile]
  \frametitle{Tradiční kompilační skripty}
  \begin{itemize}
    \item parametry pro každý jednotlivý krok se předávají kompilačnímu skriptu v závorkách
    \item základní forma pro dokument ve formátu utf8
  \end{itemize}
      \begin{priklad}
        \small 
\begin{verbatim}
htlatex jmenosouboru.tex "xhtml,charset=utf-8" 
" -cmozhtf -utf8"
\end{verbatim}
     \end{priklad}
\end{frame}

\section{Sestavovací program make4ht}
\begin{frame}
  \frametitle{Problémy, které make4ht řeší} 
  \begin{itemize}
    \item složité předávání parametrů pro htlatex a ostatní skripty
    \item pevně nastavené pořadí volání jednotlivých kroků kompilace
      \begin{itemize}
        \item TeX je vždy volán třikrát
      \end{itemize}
    \item podpora pro nástroje jako je \texttt{bibtex}, \texttt{xindy} a podobně
    \item  snadná změna parametrů konverze obrázků
    \item funkční kopírování souborů do výstupního adresáře
    \item zpracování výstupních souborů filtrovacími funkcemi v jazyce Lua, nebo externími
      programy
  \end{itemize}
\end{frame}
\begin{frame}[fragile]
  \frametitle{Obecné použití}
      \begin{priklad}
        \small
\begin{verbatim}
make4ht [přepínače pro make4ht] soubor.tex \ 
"volby pro tex4ht.sty" "přepínače pro tex4ht" \
"přepínače pro t4ht" "přepínače pro TeX"
\end{verbatim}
      \end{priklad}
\end{frame}

\begin{frame}[fragile]
  \frametitle{Výstup v kódování  UTF8}
      \begin{priklad}
\begin{verbatim}
make4ht -u soubor.tex
\end{verbatim}
\end{priklad}

\end{frame}


\begin{frame}[fragile]
  \frametitle{Přepínače}
\begin{description}
  \item[\prepinac{utf8 (-u)}] výstup v kódování UTF-8.
  \item[\prepinac{mode (-m)}] volba módu kompilace.
  \item[\prepinac{lua (-l)}] dokument bude kompilován pomocí Lua\LaTeX u.
  \item[\prepinac{config (-c)}] konfigurační soubor pro \texttt{TeX4ht}. 
  \item[\prepinac{build-file (-e)}] sestavovací soubor.
  \item[\prepinac{output-dir (-d)}] adresář, do kterého budou zkopírovány výstupní soubory.
  \item[\prepinac{shell-escape (-s)}] použije volbu \verb|--shell-escape| pro \LaTeX
  \item[\prepinac{xetex (-x)}] dokument bude kompilován pomocí Xe\LaTeX u.
  \item[\prepinac{format (-f)}] volba výstupního formátu a rozšíření.
\end{description}

\end{frame}

\begin{frame}
  \frametitle{Podporované formáty}
  \begin{itemize}
    \item html5
    \item xhtml
    \item odt
  \end{itemize}
\end{frame}

\begin{frame}[fragile]
  \frametitle{Rozšíření}
  \begin{priklad}
\begin{verbatim}
make4ht -f html5+tidy simple-example.tex
\end{verbatim}
\end{priklad}
\end{frame}
\begin{frame}
  \frametitle{Podporovaná rozšíření}
\begin{description}
  \item[latexmk\_build] využije příkaz \texttt{latexmk} pro sestavení
    dokumentu.
  \item[tidy] vyčistí HTML soubor pomocí příkazu \texttt{tidy}.
  \item[dvisvgm\_hashes] efektivní generování obrázků pomocí příkazu
    \texttt{dvisvgm}. 
  \item[common\_filters a common\_domfilters] vyčistění dokumentu pomocí filtrů. 
  \item[mathjaxnode] konverze matematického kódu ve formátu MathML do
    speciálního HTML za využití projektu \texttt{MathJax Node
    Page}. Příklad \url{https://www.kodymirus.cz/samples/mathjaxnode/maths.html}
  \item[staticsite]  vytvoří dokument použitelný pro generátory statických
    stránek.
\end{description}
\end{frame}

\begin{frame}[fragile]
  \frametitle{Konfigurační soubor .make4ht}
  \begin{priklad}
\begin{verbatim}
filter_settings "staticsite" {
  site_root = "output" 
}

Make:enable_extension("common_domfilters")
if mode=="publish" then
  Make:enable_extension("staticsite")
  Make:htlatex()
end
\end{verbatim}
\end{priklad}

\end{frame}

\begin{frame}[fragile]

  \frametitle{Kompilace}

  \begin{priklad}
\begin{verbatim}
make4ht -um publish simple-example.tex
\end{verbatim}
\end{priklad}
\end{frame}

\begin{frame}[fragile]
  \frametitle{Výsledný HTML soubor}
  \begin{priklad}
\begin{verbatim}
---
time: 1544811015
date: '2018-12-14 18:10:47'
title: 'sample'
styles:
- '2018-12-14-simple-example.css'
meta:
- charset: 'utf-8'
---
<p>Sample document</p>
\end{verbatim}
\end{priklad}
\end{frame}

\subsection{Sestavovací skripty}
\begin{frame}[fragile]

\frametitle{Ukázkový dokument}
\begin{priklad}
\begin{verbatim}
\documentclass{article}
\begin{document}
Test {\itshape háčků}
\end{document}
\end{verbatim}
\end{priklad}
\end{frame}

\begin{frame}[fragile]
  \frametitle{Výsledný HTML kód}
\begin{priklad}
\begin{verbatim}
<!--l. 4--><p class="noindent" >Test <span 
class="rm-lmri-10">h</span><span 
class="rm-lmri-10">á</span><span 
class="rm-lmri-10">čk</span><span 
class="rm-lmri-10">ů</span> </p> 
\end{verbatim}
\end{priklad}

\end{frame}


\begin{frame}[fragile]
  \frametitle{Sestavovací skript}
  \begin{priklad}
\begin{verbatim}
local domfilter = require("make4ht-domfilter")
local function domsample(dom)
  for _, par in ipairs(dom:query_selector("p")) do
    par:set_attribute("class", "mypar")
  end
  return dom
end
local process = domfilter({
  "joincharacters", 
  domsample})
Make:match("html$", process)
\end{verbatim}
  \end{priklad}
\end{frame}

\begin{frame}[fragile]
  \frametitle{Upravený HTML kód}
  \begin{priklad}
\begin{verbatim}
<!-- l. 3 --><p class='mypar'>
Test <span class='rm-lmri-10'>háčků</span> 
</p> 
\end{verbatim}
  \end{priklad}
\end{frame}

\begin{frame}[fragile]
  \frametitle{Komplikovanější sestavovací soubor}
  \begin{priklad}
\begin{verbatim}
Make:add("biber","biber ${input}")
Make:htlatex {}
Make:biber {}
Make:htlatex {}
Make:image("png$",                                                          
"dvipng -bg Transparent -T tight -o ${output}"..
"-pp ${page} ${source}")
\end{verbatim}
  \end{priklad}
\end{frame}

\section{Konfigurace tex4ht}
\begin{frame}
  \frametitle{Princip TeXové části tex4ht}
  \begin{itemize}
    \item soubor tex4ht.sty se nahrává ještě před načtením dokumentu, jeho
      nahrávání si dále řídí sám
  \item po zpracování preamble a nahrání všech balíčků se spouští .4ht soubory
    s vkládáním háčků pro dané balíčky, pokud existují
    \begin{pozor}<2>
      Protože se příkazy redefinují až na začátku dokumentu, příkazy
      volané v preamble dokumentu nejsou ještě redefinované
    \end{pozor}
\end{itemize}
\end{frame}

\begin{frame}
  \frametitle{Kdy je třeba vkládat háčky?}
  \begin{itemize}
\item konfigurace je důležitá hlavně pro příkazy se složitější strukturou nebo
      logické bloky (nadpisy, tabulky, seznamy apod.)
    \item  pokud makra staví na základních prvcích, pro které už existuje
      podpora, nemusí být třeba přidávat pro ně jejich vlastní podporu
  \end{itemize}
\end{frame}

\begin{frame}
  \frametitle{Konfigurace háčků}
  \begin{itemize}
    \item soubory výstupního formátu mohou definovat volby, které ovlivňují konfiguraci háčků
    \item konfigurace probíhá příkazem \texttt{\textbackslash Configure}

    \item po vložení háčků se nahrají jejich konfigurace v závislosti na
      výstupním formátu
    \item další konfigurace je možné vložit do .cfg souboru, který se načítá pomocí volby \texttt{-c} příkazu \texttt{make4ht}

\end{itemize}
\end{frame}

\begin{frame}[fragile]
  \frametitle{Volby balíčku tex4ht.sty}
  \begin{itemize}
    \item mohou být předány pomocí argumentů pro tex4ht
    \item nebo v .cfg souboru
  \end{itemize}
  \begin{priklad}
\begin{verbatim}
make4ht filename.tex "mathml,mathjax"
\end{verbatim}
\end{priklad}
\end{frame}

\begin{frame}
  \frametitle{Některé dostupné volby}
\begin{itemize}
  \item fn-in
  \item pic-m, pic-align
  \item svg
  \item info
  \item mathml
  \item mathjax
\end{itemize}
\end{frame}


\begin{frame}[fragile]
  \frametitle{Soukromý konfigurační soubor}
  \begin{itemize}
    \item umožňuje vkládání konfigurací pro háčky a CSS instrukce
    \item základní struktura
      \begin{verbatim}
% zde můžeme vkládat balíčky
\Preamble{xhtml,volby pro tex4ht.sty}
...
\begin{document}
...
\EndPreamble
      \end{verbatim}
  \end{itemize}
\end{frame}



\begin{frame}[fragile]
  \frametitle{Užitečné příkazy}
  \begin{itemize} 
  \item \verb|\Configure|, \verb|\ConfigureEnv|, \verb|\ConfigureList|
  \item \verb|\HCode|, \verb|\Css|, \verb|\Hnewline|
  \item \verb|\EndP|, \verb|\IgnorePar|
  \item \verb|\Picture+|, \verb|\Picture*|
  \item \verb|\NoFonts|
\end{itemize}
\end{frame}

\begin{frame}[fragile]
  \frametitle{Ukázka základní konfigurace}
  \begin{priklad}
\begin{verbatim}
\Configure{textit}{\HCode{<em>}\NoFonts}
{\EndNoFonts\HCode{</em>}}
\end{verbatim}
  \end{priklad}
\end{frame}


\begin{frame}[fragile]
  \frametitle{Problematika odstavců}
  \begin{priklad}
\begin{verbatim}
\ConfigureEnv{rightaligned}
{\HCode{<section class="right">}}
{\HCode{</section>}}{}{}
\end{verbatim}
\end{priklad}
Výsledný HTML kód je chybný
\begin{priklad}
\begin{verbatim}
<p class="indent" ><section class="right">
\end{verbatim}
\end{priklad}
\end{frame}

\begin{frame}[fragile]
  \frametitle{Správné řešení}
  \begin{priklad}
\begin{verbatim}
\ConfigureEnv{rightaligned}
{\ifvmode\IgnorePar\fi\EndP%
\HCode{<section class="right">}\par}
{\ifvmode\IgnorePar\fi\EndP%
\HCode{</section>}}{}{}
\end{verbatim}
  \end{priklad}
  HTML kód

  \begin{verbatim}
<section class="right">
<!--l. 9--><p class="indent" >
  \end{verbatim}
\end{frame}

\begin{frame}[fragile]
  \frametitle{Obrázky}
  \begin{priklad}
\begin{verbatim}
\ConfigureEnv{topicture}{\Picture*{}}
{\EndPicture}{}{}
\end{verbatim}
\end{priklad}
\end{frame}

\begin{frame}[fragile]
  \frametitle{Kompletní konfigurační soubor}
  \begin{priklad}
    \small
    \begin{verbatim}
\Preamble{xhtml}
\Configure{textit}{\HCode{<em>}\NoFonts}
{\EndNoFonts\HCode{</em>}}
\ConfigureEnv{rightaligned}
{\ifvmode\IgnorePar\fi\EndP
\HCode{<section class="right">}\par}
{\ifvmode\IgnorePar\fi\EndP
\HCode{</section>}}{}{}
\ConfigureEnv{topicture}
{\Picture*{}}{\EndPicture}{}{}
\Css{.right{text-align:right;display:block;}}
\begin{document}
\EndPreamble
  \end{verbatim}
  \end{priklad}
\end{frame}

\begin{frame}[fragile]
\frametitle{Reálný příklad}
\begin{verbatim}
\Preamble{xhtml,mathml,2,sec-filename,fn-in,fonts}
% external stylesheet
\Configure{AddCss}{style.css}
\end{verbatim}
\end{frame}

\begin{frame}[fragile]
  \frametitle{Konfigurace mini obsahu na jednotlivých stránkách}
  \small
  \begin{verbatim}
\Configure{crosslinks+}{%
  \bgroup
  \Configure{tableofcontents}
  {\IgnorePar\EndP
  \HCode{<nav class="TOC">}\IgnorePar}
  {\HCode{\Hnewline}}
  {\IgnorePar\HCode{</nav>\Hnewline}\ShowPar}{}{}%
  \TableOfContents[chapter,section,subsection]
  \egroup
  \ifvmode\IgnorePar\fi\EndP%
  \HCode{<article class="main-content">\Hnewline
  <nav class="crosslinks-top">}}
  {\HCode{</nav>\Hnewline}}
{\ifvmode\IgnorePar\fi\EndP%
  \HCode{<nav class="crosslinks-bottom">}}
  {\HCode{</nav>}}{}{}
  \end{verbatim}
\end{frame}

\begin{frame}[fragile]
  \frametitle{Další nastavení}
  \begin{verbatim}
% uzavřít <article>, který byl otevřen v 
% \Configure{crosslinks+}
\Configure{@/BODY}
{\ifvmode\IgnorePar\fi\EndP\HCode{</article>}}

% Ukázat odkazy pouze na předešlou, 
% hlavní a násladující stránku
\Configure{crosslinks*}{prev}{up}{next}{}
\begin{document}
\EndPreamble
\end{verbatim}
\end{frame}

\section{Přidání podpory pro nové balíčky}

\begin{frame}
  \frametitle{Základní postup}
  \begin{itemize}
    \item vytvořit soubor pojmenovaný \textit{název balíčku + .4ht}
    \item konfigurovat balíček v \texttt{4ht} souboru výstupního formátu
  \end{itemize}
\end{frame}


\begin{frame}[fragile]
  \frametitle{Ukázkový balíček \texttt{custom.sty}}
  \begin{verbatim}
\ProvidesPackage{custom}
\newcommand\custom[1]{\bgroup\itshape#1\egroup}
\endinput
  \end{verbatim}
\end{frame}

\begin{frame}[fragile]
  \frametitle{Vložení háčků v souboru \texttt{custom.4ht}}
  \begin{verbatim}
\NewConfigure{custom}{2}
\pend:defI\custom{\a:custom}
\append:defI\custom{\b:custom}
\Hinput{custom}
  \end{verbatim}
\end{frame}


\begin{frame}[fragile]
  \frametitle{Vlastní výstupní formát \texttt{myoutput.4ht}}
  \begin{verbatim}
\exit:ifnot{custom} 
\ConfigureHinput{custom} 
\Configure{custom}
{\HCode{<span class="custom">}\NoFonts}
{\EndNoFonts\HCode{</span>}}
\endinput\empty\empty\empty\empty\empty\empty 
%%%%%%%%%%%%%%%%%%%%%%%%%%%%%%%%%%%%%%%%%%%%%%
\endinput 
  \end{verbatim}
\end{frame}

\begin{frame}[fragile]
  \frametitle{Definice vlastního formátu \texttt{tex4ht.usr}}
  \begin{verbatim}
\Configure{myoutput}{% 
   \Hinclude[*]{html4.4ht}%
   \Hinclude[*]{unicode.4ht}%
   \Hinclude[*]{html5.4ht}%
   \Hinclude[*]{myoutput.4ht}% 
} 
  \end{verbatim}
\end{frame}

\begin{frame}[fragile]
  \frametitle{Kompilace vlastního formátu}
  \begin{verbatim}
make4ht -um draft sample.tex myoutput
  \end{verbatim}
\end{frame}

\begin{frame}
\frametitle{To je vše}
\begin{block}{}
Děkuji za pozornost
\end{block}
\end{frame}

\end{document}
