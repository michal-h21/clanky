\def\myminus{-}
\documentclass{csbulletin}
\usepackage{luatex85}
\usepackage{fontspec}
\usepackage{hologo}
\usepackage{luavlna}
\usepackage{url}

\preventsingledebugon

\begin{document}
\title{\texttt{luavlna}: \texttt{vlna} pro formáty \hologo{LuaTeX}u}
\EnglishTitle{\texttt{luavlna}: \texttt{vlna} for \hologo{LuaTeX} formats}
\author{Michal Hoftich}
\maketitle
\begin{abstract}
Sem přijde abstrakt
\end{abstract}

\section{Úvod}
Podle českých typografických norem by jednopísmenné předložky neměly stát na 
konci řádků. V \TeX u se tento problém řeší nahrazením mezery znakem \verb|~|,
který vkládá nezalomitelnou mezeru. Protože se na toto pravidlo snadno zapomíná,
existují různé pomůcky, které vkládají nezalomitelnou mezeru automaticky.

Klasickým nástrojem je program \verb|vlna|, který provádí nahrazení přímo na
úrovni zdrojového kódu. Později vznikly další projekty, které nemodifikují přímo
zdrojový kód, ale modifikují tok řetězců nebo tokenů při zpracování dokumentu 
\TeX em. Jedná se o \textsc{encxvlna} pro Enc\TeX\ a X\raisebox{\myminus.5ex}{Ǝ}\kern\myminus.3ex Vlna pro \hologo{XeTeX}.
% Nedaří se mi použít záporné hodnoty pro \raisebox, je znak '-' aktivní?

Protože \hologo{LuaTeX} nabízí snadné zpracování datových struktur \TeX u 
(boxy, nódy, fonty, apod.), jeví se jako logický krok využití těchto možností
k vytvoření nové verze \verb|vlny|, která by využívala těchto schopností. 

Jako odpověď na otázku možnosti automatického vkládání nezalomitelných mezer 
na serveru TeX.sx\footnote{\url{http://tex.stackexchange.com/q/27780/2891}}, 
Patrick Gundlach předvedl řešení využívající nódy zpracovávající callbacky 
\hologo{LuaTeX}u. Toto řešení se později stalo součástí balíčku\footnote{%
\url{http://www.ctan.org/pkg/impnattypo}} a také moje řešení z něho vychází.

\section{Callbacky a nódy v \hologo{LuaTeX}u}

Callbacky jsou funkce, které jsou volány při různých událostech v průběhu
zpracování dokumentu. Velkou skupinou callbacků jsou callbacky zpracovávající 
seznamy nódů, datovou strukturu obsahující znaky, mezery a ostatní, vzniklou po
zpracování tokenů. Výhodou je, že tato struktura obsahuje již finální informaci,
která bude vysázená v dokumentu, po expanzi všech maker. Nódy samotné jsou 
objekty, které obsahují různá pole v závislosti na typu nódu.

\iffalse
Pojďme se podívat na praktickou ukázku, která vypíše do terminálu v průběhu 
kompilace text dokumentu. Využijeme k tomu callback \verb|pre_linebreak_filter|,
který zpracovává seznam nódů před zalomením do odstavců.

\begin{verbatim}
\uselanguage{czech}
\input luaotfload.sty
\font\normal={Linux Libertine O} at 12pt
\normal
% vložíme callback
\directlua{%
	local clb = require "sample-callback"
	luatexbase.add_to_callback("pre_linebreak_filter",clb, "sample callback")
}

Příliš žluťoučký kůň \hbox{úpěl}. \uselanguage{british}English sentence.

\bye
	
\end{verbatim}

Tato ukázka je určená pro čistý \hologo{LuaTeX}, callback je vkládán v 
příkazu \verb|directlua|, jeho samotný kód je umístěn v souboru 
\verb|sample-callback|:

\begin{verbatim}
local glyph_id = node.id("glyph")
local glue_id  = node.id("glue")

-- Převede pole nódu char na znak
local char = unicode.utf8.char 

local function mycallback(head)
	local text = {}
	for n in node.traverse(head) do
		if n.id == glyph_id then
			table.insert(text, char(n.char))
		elseif n.id == glue_id then
			table.insert(text, " ")
		end
	end
	print(table.concat(text, ", " ))
	return true
end

return mycallback
\end{verbatim}	

\fi

Protože potřebujeme zpracovat seznam nódů před zalomením do odstavců, můžeme
využít callback pojmenovaný \verb|pre_linebreak_filter|.

\begin{summary}
	Anglický abstrakt

\end{summary}
\end{document}
