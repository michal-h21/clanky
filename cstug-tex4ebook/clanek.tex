\documentclass{csbulletin}
\usepackage[T1]{fontenc}
\usepackage[utf8]{inputenc}
\usepackage[czech]{babel}
\usepackage{url}
\usepackage{metalogo}
\newcommand\footlink[1]{\footnote{\url{#1}}}
\begin{document}
% # osnova

% ## úvod do ebooků
\section{Elektronické knihy}

Elektronické knihy (e-booky) se rozvíjejí především od 70.~let 20.~století, kdy
byl spuštěn projekt Guttenberg, jehož cílem je digitalizace knih a textů, na
které se nevztahují autorská práva. Prudký rozvoj nastal od 90. let, kdy se
objevily osobní digitální asistenty (PDA) a masověji se rozšířily osobní
počítače. Zpočátku je tvořili především nadšenci, komerční vydavatelství pouze
pomalu ztrácela nedůvěru. V roce 2004 se objevila první čtečka s displejem s
elektronickým inkoustem (e-ink), o tři roky později internetový obchod Amazon
představil svojí čtečku Kindle a nastal velký komerční rozvoj.

Hlavními vlastnostmi e-booků jsou flexibilita zobrazení, neboť je třeba, aby
šly zobrazit na řadě zařízení s různou velikostí a vlastnostmi displeje,
možnost upravit vlastnosti zobrazení, jako je velikost nebo barva písma podle
potřeb konkrétního čtenáře. Některé čtečky také umožňují předčítání textu
technologií text to speech.

V průběhu vývoje se objevilo množství formátů e-booků, v dnešní době jsou
nejrozšířenějšími Epub, z něho vycházející Epub~3 a Mobi, používaný Amazonem.
% Formát PDF se často nepovažuje za e-bookový formát, protože 
Tyto formáty v podstatě vychází z formátů HTML a CSS, běžně používaných na WWW
stránkách, a pomocných metadat. Formát Mobi se vytváří kompilací Epubu
programem Kindlegen, který Amazon poskytuje pro Windows, Mac OS i Linux. 

Výhodou Epubu je jeho široká podpora nejrůznější škálou zařízení a programů,
které ovšem často podporují pouze některé jeho vlastnosti, často mají problém
například s CSS styly, takže formátování e-booku může být zcela rozdílné, než
autor zamýšlel. Nejlepší využití Epubu je pro jednoduchý text bez složitějšího
formátování, například pro beletrii nebo literaturu faktu. 

To samé platí v podstatě i o formátu Mobi, který vzniká konverzí z Epub. V
podstatě se jedná o dva formáty v jednom, neboť Kindlegen vytváří verzi pro
novější i starší verzi svých zařízení. Amazon pro autory připravil dokument
\textit{Kindle design guidelines}, kde konkrétně specifikuje, jakým způsobem by
měl být zdrojový Epub soubor připraven, aby konverze proběhla nejlepším
způsobem.

Protože formát Epub vznikal živelně a jeho implementace v čtecích zařízeních je
nejednotná, organizace IPDF\footnote{The International Digital Publishing
Forum} vytvořila standard Epub~3. Ten by měl mimo jiné zlepšit podporu pro
technické publikace nebo pro asistenční služby pro čtenáře se specifickými
potřebami. Tento formát bude tedy pro uživatele \TeX u patrně nejzajímavější,
ovšem je třeba zmínit, že přestože je tento formát standardizovaný, čtecí
zařízení zdaleka nepodporují všechny jeho vlastnosti. V současné době vzniká
jeho nová verze, Epub~3.1, která obsahuje opět poměrně zásadní změny, ovšem
zatím nebyla představena finální verze, proto se jím nebudeme dále zabývat.

Užitečné vlastnosti Epub~3 pro technické publikace jsou podpora MathML
pro sazbu matematiky\footnote{V ostatních formátech je třeba používat obrázky,
což je především pro matematiku vloženou v textu nevhodné řešení}, podpora SVG pro vektorové obrázky, podpora JavaScriptu,
která umožňuje tvorbu interaktivních prvků, vkládání fontů do dokumentu.
Významnou změnou jsou  nové HTML elementy pro značkování logické struktury
dokumentu (například poznámky pod čarou, bibliografie, rejstřík). Formát SMIL
umožňuje vkládání multimediálního obsahu.

Patrně nejzajímavější ukázkou možností Epub~3, která v současnosti existuje, je
projekt Aeneas\footlink{https://www.readbeyond.it/aeneas/}, který umožňuje
synchronizovat čtenou verzi knihy s jejím textem. Takto připravené e-booky
lze číst v aplikaci,
Menestrello\footlink{https://www.readbeyond.it/menestrello/index.html}, která
zvýrazňuje právě předčítaný text. Je to vynikající pomůcka například pro
procvičování výuky jazyků.



\section{Konverze \TeX u na e-book}

Kvůli stále rozšířenějšímu čtení textů z elektronických zařízení, ať již jsou
to PC, mobilní telefony, tablety nebo e-ink čtečky, vzniká potřeba konverze z
\TeX u do do e-booků.  Všechny zmíněné formáty jsou v podstatě balíčky WWW
stránek s přidanými metadaty, takže tato úloha se v podstatě skládá z konverze
z \TeX u do HTML, přidání nezbytných metadat a zabalení výsledných souborů do
e-bookového formátu.

Pro konverzi \TeX u do HTML se dá použít řada nástrojů, nejrozšířenějšími jsou
LaTeX2HTML\footlink{http://www.latex2html.org/},
LaTeXML\footlink{http://dlmf.nist.gov/LaTeXML/},
Pandoc\footlink{http://pandoc.org/} a
TeX4ht\footlink{http://www.tug.org/tex4ht/}. První tři z nich jsou založené na
zpracování \TeX ového dokumentu externím programem, který se snaží rozpoznat
použitá makra a převést je na HTML. LaTeXML je z nich nejpokročilejší, neboť 
emuluje \TeX ovou expanzi maker a podporuje tudíž i uživatelská makra a
balíčky. Zbylé dva systémy podporují především základní příkazy \LaTeX u a se
složitějšími dokumenty mohou mít problém. Všechny podporují pouze \LaTeX,
nikoli jiné formáty. 

TeX4ht je sada maker, která se automaticky načítá do dokumentu a upravuje
definici jednotlivých příkazů tak, aby bylo možné vkládat formátovací značky
podporovaných výstupních formátů (kromě HTML je to především OpenDocument
Format, TEI nebo Docbook). Díky tomu, že pro překlad je používaný samotný \TeX,
neměl by být problém s užíváním uživatelských maker a balíčků. Kromě \LaTeX u
podporuje i jiné formáty, například Plain, ovšem u nich může být třeba upravit
dokument. Pro překlad je možné použít současné nejpoužívanější \TeX ové enginy,
tedy pdf\TeX, \XeTeX a \LuaTeX. Systém vytvářel Eitan Gurari, po jeho smrti
vývoj převzal Karl Berry, CV Radhakrishnan a autor článku. Právě TeX4ht je
použit jako základ pro TeX4ebook\footlink{https://www.ctan.org/pkg/tex4ebook},
systém pro konverzi \TeX u na e-booky.

\section{TeX4ebook}

TeX4ebook je systém pro konverzi \TeX u do formátů Epub, Epub~3 a Mobi, hlavní
důraz je kladený na Epub~3. Hlavní výhodou oproti pouhé konverzi \TeX u do HTML
a jeho následné konverzi na e-book externími nástroji je možnost získání
různých metadat přímo ze zpracovávaného dokumentu. Systém emuluje základní
postup kompilace používaný TeX4ht a přidává několik kroků navíc. Celý postup je
konfigurovatelný s použitím sestavovacího skriptu.

Základní postup je shodný s TeX4ht a sestává ze tří kroků:

\begin{enumerate}
  \item Kompilace dokumentu \TeX em. Ještě před načtením dokumentu je nahrán
    balíček \texttt{tex4ht.sty}, který řídí další postup zpracování dokumentu.
    Pro každý načtený balíček se testuje existence souboru s příponou
    \texttt{.4ht}, který, pokud existuje, je načten po skončení preamble
    dokumentu. V tomto konfiguračním souboru se mohou předefinovávat makra a
    vkládají se do nich takzvané \textit{háčky}. Po nahrání všech \texttt{.4ht}
    souborů pro použité balíčky jsou nahrány soubory s konfigurací pro daný
    výstupní formát. Obsahují instrukce, které jsou vloženy do háčků
    definovaných v předešlém kroku. Háčky je také možné konfigurovat pomocí
    konfiguračního souboru, který bude popsaný dále. 

    Kompilace je spouštěna několikrát, protože jsou využívány pomocné soubory
    pro tvorbu hypertextových odkazů. Obyčejně jsou třeba tři kompilace, aby
    všechny odkazy fungovaly správně, ve výjimečných případech může být třeba i
    více kompilací. 

  \item Zpracování DVI souboru příkazem \texttt{tex4ht}. Tento příkaz vytváří
    výstupní soubory, převádí různá vstupní kódování na \texttt{UTF-8} a
    vytváří dva pomocné soubory: \texttt{.idv} soubor je speciální soubor ve
    formátu DVI, který obsahuje stránky, které mají být převedeny na obrázky.
    Jedná se například o prostředí \texttt{picture} LaTeXu, nebo matematiku,
    pokud není převáděna do formátu MathML. \texttt{.lg} soubor obsahuje seznam
    výstupních souborů, CSS instrukce, a instrukce pro kompilaci jednotlivých
    stránek v \texttt{.idv} souboru na obrázky.
  \item Zpracování \texttt{.lg} souboru příkazem \texttt{t4ht}. V tomto kroku
    je vytvořen CSS soubor, dochází ke konverzi obrázků a výstupní soubory
    mohou být předány externím programům k dalšímu zpracování.
\end{enumerate} 

TeX4ht používá pro kompilaci řadu skriptů, které se liší použitým \TeX ovým
formátem, enginem a výstupním formátem. Všechny provádí tři kompilační kroky
popsané výše a předávají jednotlivým příkazům argumenty v uvozovkách.
Například pro vytvoření dokumentu ve formátu XHTML a kódování \texttt{UTF-8} je
možné použít následující příkaz: 
  
\begin{verbatim}
htlatex filename.tex "xhtml,charset=utf-8" " -utf8 -cunihtf"
\end{verbatim}

Tento postup není úplně uživatelsky přívětivý, proto TeX4ebook umožňuje
použít pro základní konfiguraci přepínače, známé z běžných příkazů užívaných
například v Linuxu.

\begin{verbatim}
tex4ebook -f epub3 filename.tex
\end{verbatim}

Tento příkaz vytvoří soubor \texttt{filename.epub} ve formátu Epub~3, kódování
\texttt{UTF-8} je použito automaticky. 

Hlavní rozdíl TeX4ebook oproti TeX4ht spočívá ve třetím kompilačním kroku.
Příkaz \texttt{t4ht} je použit pouze pro vytvoření CSS souboru, konverzi
obrázků a spouštění externích příkazů řídí TeX4ebook sám. Kromě toho je možné
spouštět příkazy mezi jednotlivými kompilacemi \TeX u, například po první
kompilaci a tvorbě pomocných souborů je možné spustit příkazy pro tvorbu
rejstříku a bibliografie. 

Postup kompilace je možné řídit pomocí sestavovacích skriptů pro
Make4ht\footlink{https://www.ctan.org/pkg/make4ht}. To je sestavovací program
pro TeX4ht, který původně vznikal jako součást TeX4ebook, ale postupně se
vyvinul do samostané aplikace a knihovny, která nahrazuje skripty používané pro
kompilaci TeX4ht.

Sestavovací skripty jsou programy v jazyce Lua:

\begin{verbatim}
local filter = require "make4ht-filter"
local process = filter{"hruletohr"}
Make:add("biber","biber ${input}")
Make:htlatex{}
Make:biber {}
Make:htlatex {}
Make:image("png$",                                                          
"dvipng -bg Transparent -T tight -o ${output}"..
"-pp ${page} ${source}")
Make:match("html$",process)
Make:match("html$",
"tidy -m -utf8 -asxhtml -q -i ${filename}")
\end{verbatim}


Pomocí příkazu \verb|Make:add("biber", "biber ${input}")| definujeme příkaz
\verb|Make:biber|, který můžeme použít  pro tvorbu bibliografie s balíčkem
Biblatex. Druhý argument příkazu \verb|Make:add| je šablona s příkazem který se
spustí, proměnné definované Make4ht se vkládají pomocí konstruktu
\verb|${jméno}|, proměnná \texttt{input} obsahuje název zpracovávaného souboru.

Argumenty ve složených závorkách pro příkazy \texttt{Make:htlatex} a
\texttt{Make:biber} jsou tabulky, ve kterých můžete nastavit hodnoty
proměnných užívaných v šablonách, pokud byste například chtěli použít
bibliografii z jiného souboru, lze použít:

\begin{verbatim}
Make:biber {input = "jinysoubor"}
\end{verbatim}

Příkazy \texttt{Make:image} a \texttt{Make:match} se spouští na výstupních
souborech. Jejich  prvním argument je regulární výraz, kterým se testují jména
souborů, druhým je příkaz, který se spustí, pokud jméno souboru vyhovuje
hledanému vzoru. Příkaz může být jednak šablona s programem, nebo Lua funkce.

Při konverzi obrázků se používají proměnné, které nejsou k dispozici v
kompilačních příkazech, konkrétně \texttt{source} obsahuje jméno \texttt{.idv}
souboru, \texttt{output} je název obrázku a \texttt{page} udává číslo stránky v
\texttt{.idv} souboru. 

Příkaz \texttt{Make:match} umožňuje upravovat výstupní soubory. Proměnná
\texttt{filename} obsahuje název výstupního souboru. V našem případě je tento
příkaz spuštěn dvakrát, první příklad ukazuje použití filtrů, které jsou funkce
v jazyce Lua. V tomto případě se jedná o filtr, který napravuje problém
způsobený příkazem \verb|\hrule|, jenž TeX4ht nemůže redefinovat a v HTML
souboru se projeví jako řada znaků \verb|_|. Jako argument pro příkaz
\texttt{filter} lze zadat více filtrů, kromě předdefinovaných jako je
\texttt{hruletohr}, lze zadat i vlastní funkce:

\begin{verbatim}
local changea = function(s) return s:gsub("a","z") end
local process = filter{"hruletohr", changea}
\end{verbatim}

Tento příklad není úplně užitečný, neboť změní všechna písmena \uv{a} ve
zpracovávaném dokumentu na \uv{z}, včetně HTML tagů, ovšem jako ukázka snad
postačí.

Pokud sestavovací soubor pojmenujete stejně jako název dokumentu s příponou
\texttt{.mk4}, bude spouštěn automaticky, jinak ho lze zadat jako parametr
\texttt{-e} pro TeX4ebook:

\begin{verbatim}
tex4ebook -f epub3 -e buildfile.mk4 filename.tex
\end{verbatim}

\section{Konfigurační soubor pro TeX4ht}

Pro vložení vlastních HTML tagů do konfigurovatelných háčků, nebo vlastních CSS
instrukcí můžeme použít konfigurační soubor pro TeX4ht. Možnosti konfigurace si
ukážeme na jednoduchém \TeX ovém souboru:

\begin{verbatim}
\documentclass{article}
\usepackage[T1]{fontenc}
\usepackage[utf8]{inputenc}
\usepackage[czech]{babel}
\begin{document}
Příliš žluťoučký kůň \textit{úpěl} 
\textbf{ďábelské ódy} 
\end{document}
\end{verbatim}

Po kompilaci se vygeneruje HTML soubor s následujícím obsahem:


\begin{verbatim}
<p class="noindent" >Příliš žluťoučký kůň <span 
class="ecti-1000">úp</span><span 
class="ecti-1000">ěl </span><span 
class="ecbx-1000">ď</span><span 
class="ecbx-1000">ábelsk</span><span 
class="ecbx-1000">é </span><span 
class="ecbx-1000">ódy </span>
\end{verbatim}

Vygenerované soubory jsou ponechány i v adresáři s \TeX ovým dokumentem, lze
je tedy snadno zkontrolovat ve WWW prohlížeči.
Slova která byla zvýrazněna jiným, než základním řezem písma, byla rozdělena do
několika HTML elementů. Ve vykresleném dokumentu se text zobrazí v pořádku,
nicméně měli bychom tento problém vyřešit. Jedná se o chybu při konverzi
\texttt{DVI} souboru příkazem \texttt{tex4ht}, který se snaží přidat
formátování podle písma detekovaného v \texttt{DVI} souboru. Díky této
funkcionalitě není třeba vytvářet konfigurace pro všechna uživatelská makra,
základní textové formátování, jako jsou řezy a velikost písma, je zachováno.
Pokud však TeX4ht narazí na znak s diakritikou, uzavře aktuální element,
přestože nedochází ke změně písma. Tento problém lze vyřešit jednoduchou
konfigurací pro dotčená makra v konfiguračním souboru, pojmenovaném například
\texttt{myconfig.cfg}:


\begin{verbatim}
\Preamble{xhtml}
\Configure{textbf}{\NoFonts\HCode{<strong>}}
{\HCode{</strong>}\EndNoFonts}
\Configure{textit}{\NoFonts\HCode{<em>}}
{\HCode{</em>}\EndNoFonts}
\begin{document}
\EndPreamble
\end{verbatim}

Konfigurační soubor můžeme načíst volbou \texttt{-c}:

\begin{verbatim}
tex4ebook -f epub3 -e buildfile.mk4 -c myconfig.cfg filename.tex
\end{verbatim}

Výsledný HTML kód již vypadá mnohem lépe.

  \begin{verbatim}
<p class="noindent" >Příliš žluťoučký kůň 
<em>úpěl</em> <strong>ďábelské ódy</strong> 
  \end{verbatim}

Konfigurační soubory pro TeX4ht mají pevnou základní strukturu:


      \begin{verbatim}
% zde můžeme vkládat balíčky
\Preamble{xhtml,volby pro tex4ht.sty}
...
% Konfigurace háčků a kaskádových stylů 
\begin{document}
...
\EndPreamble
      \end{verbatim}

Před příkazem \verb|\Preamble| můžeme nahrávat dodatečné balíčky pomocí příkazu
\verb|\RequirePackage|. Příkaz \verb|\Preamble| umožňuje specifikovat volby pro
které ovlivňují nahrávání konfigurací. Jsou oddělené čárkami a povinná je
minimálně volba \texttt{html} nebo \texttt{xhtml}, která musí být na první
pozici. Pro Epub je třeba používat volbu druhou z nich. 

Po příkazu \verb|\Preamble| můžete přidávat konfigurace pro
háčky, kaskádové styly, nebo měnit definice příkazů definovaných v preamble
dokumentu, ať již ve vkládaných balíčcích, nebo přímo v dokumentu.
Použití \verb|\begin{document}| je povinné, definice, které po něm následují
jsou spuštěny až na začátku dokumentu. Konfigurace končí příkazem
\verb|\EndPreamble|.

V našem příkladu užíváme dvě konfigurace, pro příkazy \verb|textbf| a
\verb|textit|. Příkaz \verb|\Configure| má variabilní počet argumentů, pro
jednotlivé konfigurace je třeba konzultovat dokumentaci\footnote{Dokumentace
  není úplně silnou stránkou projektu TeX4ht. Základní informace o konfiguraci
  se nachází na WWW stránkách projektu, detailnější informace můžete získat z
  dokumentace vygenerované ze zdrojových kódů TeX4ht, umístěných na stránkách
\url{http://michal-h21.github.io/src4ht/}. Jedná se především o dokumenty
\texttt{tex4ht-info} a \texttt{tex4ht-html4}.}. 

V tomto konkrétním případě užívá příkaz \verb|\Configure| dva argumenty, jedná
se o kód, který se vloží na začátek a na konec konfigurovaného textu. Příkazy
\verb|\NoFonts| a \verb|\EndNoFonts| zakáží a povolí zpracování fontů z
\texttt{DVI} souboru, čímž vyřešíme problém s rozdělenými slovy. Příkaz
\verb|\HCode| vkládá HTML kód do dokumentu.




Výsledný e-book vypadá poměrně stroze, pokud chcete vylepšit jeho vzhled, je
možné použít kaskádové styly. Pro jednoduché úpravy je určený příkaz
\verb|\Css|, který lze použít v konfiguračním souboru:

\begin{verbatim}
\Preamble{xhtml
% konfigurace písem vynechána
\Css{em{color:blue;}}
\begin{document}
\end{verbatim}

Tímto příkazem obarvíme text v kurzívě modrou barvou. 

Pro složitější změny je lepší vložit kompletní CSS styl. Existuje řada
projektů stylů, které se zaměřují na dobrou typografii na webu. Můžeme využít
například Scale.css\footlink{https://github.com/viljamis/Scale}, který vytvořil
finský designér  Viljami Salminen. Tento styl je responzivní, což znamená že
velikost jednotlivých prvků dokumentu se přizpůsobuje velikosti zobrazované
plochy, tak aby byla vždy optimální z hlediska dobré čitelnosti. 

Pro vložení externího stylu můžeme využít příkaz \verb|\AddCss|, který
poskytuje balíček \texttt{include4ht}. Tento balíček je jeden z balíčků, které
tvoří projekt \texttt{helpers4ht}. Ten má za cíl zjednodušit práci se systémem
TeX4ht a obsahuje balíčky pro vkládání kaskádových stylů, JavaScriptových
programů, tvorbu rejstříků a další. Balíček pro vkládání fontů můžeme použít
pro vložení fontů do našeho e-booku.

Epub~3 podporuje vkládání písem ve formátu OpenType nebo jeho zjednodušené
verzi určené pro web, WOFF. Je vhodné využívat OpenSource písma, vložením
komerčního písma bychom se mohli dopustit jeho ilegálního šíření.  Vhodná písma
jsou například EB Garamond, Linux Libertine, nebo Latin Modern. Pro každý řez
písma je třeba stáhnout odpovídající OpenType soubor a umístit ho do adresáře s
dokumentem, stejně tak je třeba umístit zde externí kaskádové styly. Upravený
konfigurační soubor:


\begin{verbatim}
\RequirePackage{include4ht}
\RequirePackage{addfont4ht}
\Preamble{xhtml}
\AddCss{scale.css}
...
\NormalFont{EBGaramond}{EBGaramond12-Regular.woff}
\BoldFont{EBGaramond}{EBGaramond12-Italic.woff}
\ItalicFont{EBGaramond}{EBGaramond12-Italic.woff}
\Configure{@HEAD}{\HCode{<style type='text/css' >
    \Hnewline body{font-family:rmfamily, 
      "EBGaramond", sans-serif;}\Hnewline
    </style>}}
\begin{document}
\EndPreamble
\end{verbatim}

Příkazy \verb|\NormalFont|, \verb|\BoldFont| a \verb|\ItalicFont| vloží fonty.
První parametr určuje jméno rodiny, na které se dále odkazujeme. Příkaz
\verb|\Configure{@HEAD}| vkládá kód do hlavičky HTML dokumentu. V tomto případě
je použit pro vložení kaskádového stylu, který vybírá námi definovanou rodinu
\textit{EBGaramond} jako hlavní písmo dokumentu.

\section{Matematika a e-booky}

Velký problém e-booků je matematika, stejně jako na webu, kde zobrazování
matematiky řeší formát MathML. Alespoň teoreticky, protože jeho podpora ve WWW
prohlížečích je špatná, všechny ho podporují pouze částečně. 

Formát MathML je součástí specifikace Epub~3, teoreticky by ho tedy měla čtecí
zařízení podporovat, realita je ovšem odlišná\footnote{Tabulku shrnující
  podporu lze nalézt na stránkách knihovny MathJax:
\url{http://docs.mathjax.org/en/latest/misc/epub.html}}. Pokud MathML
podporují, je to za pomoci knihovny MathJax. Ta dokáže výborně vykreslit
matematiku na WWW stránkách, ovšem je poměrně náročná na výkon a na tabletech
nebo mobilních telefonech trvá delší chvíli, než se stránky obsahující
matematiku vykreslí.

V případě ostatních formátů je situace ještě horší, v případě Epub i Mobi lze
matematiku zobrazit pouze za pomoci obrázků, což pro TeX4ht není problém,
naopak je to jeho výchozí nastavení. Problémem je, že tyto obrázky nebudou
sedět na účaří a velikost a typ písma nebudou odpovídat okolnímu textu. Zatím
jsem nenalezl uspokojivé a univerzální řešení těchto problémů, proto doporučuji
pro matematické e-booky používat Epub~3  podporou MathML, jehož podpora ve
čtecích zařízeních se může zlepšit.

Podpora MathML se může zapnout pomocí přepínače \texttt{mathml} umístěného v
příkazu \verb|\Preamble| v konfiguračním souboru, nebo ve druhém argumentu
příkazu \texttt{tex4ebook}:

\begin{verbatim}
tex4ebook -f epub3 filename mathml
\end{verbatim}

\section{Konfigurace vlastních balíčků}

Příkazem \verb|\Configure| můžeme konfigurovat pouze příkazy, do kterých byly
pro toto použití vložené háčky. Pokud chceme přidat podporu pro nový balíček,
je třeba vytvořit pro něj \texttt{.4ht} soubor. Jeho tvorbu si ukážeme na
jednoduchém příkladu.

Mějme balíček pro tvorbu vzdělávacích dokumentů, který umožňuje tvorbu dvou
verzí dokumentu -- pro studenty a pro učitele, s poznámkami, které se ve
studentské verzi nezobrazují. Balíček bude mít přepínač, kterým se bude vybírat
mezi zobrazeními, pro každou verzi bude vytvořen soubor, který bude sloužit pro
kompilaci a který bude vkládat samotný text dokumentu.

Budeme také využívat rozšířených metadat, která poskytuje formát Epub~3.
Vzniklo několik profilů, které specifikují metadata podle určení dokumentu.
Pro tvorbu vzdělávacích dokumentů vznikl profil \textit{Epub for
Education}\footlink{http://www.idpf.org/epub/profiles/edu/spec/}, který
využijeme v našem dokumentu. 

Balíček \texttt{edupub.sty} je poměrně prostý:


\begin{verbatim}
\ProvidesPackage{edupub}
\RequirePackage{kvoptions}
\RequirePackage{etoolbox}
\newbool{teacher}
\boolfalse{teacher}
\newcommand\teacherinfo[1]{}
\DeclareVoidOption{teacher}{%
  \renewcommand\teacherinfo[1]{%
    \edupub@print@teacherinfo{##1}}
  \booltrue{teacher}}
\newcommand\edupub@print@teacherinfo[1]{#1}
\ProcessKeyvalOptions*
\end{verbatim}

Definuje pouze příkaz \verb|\teacherinfo|, který vypíše poznámky pro učitele
pokud je balíček nahrán s volbou \texttt{teacher}. Nyní vytvoříme konfigurační
soubor pro TeX4ht, \texttt{edupub.4ht}:


\begin{verbatim}
\ifbool{teacher}{%
  \Configure{OpfMetadata}
  {\HCode{<dc:type>teacher-edition</dc:type>}}
}{}%
\NewConfigure{teacherinfo}{2}

\let\old:teacherinfo\edupub@print@teacherinfo

\renewcommand\edupub@print@teacherinfo[1]{
  \a:teacherinfo
  \old:teacherinfo{#1}
  \b:teacherinfo
}
\Configure{teacherinfo}
{\HCode{<span epub:type="answer">}}
{\HCode{</span>}}
\end{verbatim}

Konfigurace \verb|\Configure{OpfMetadata}| vloží informace do souboru s
metadaty pro dokument. V tomto případě označí verzi dokumentu pro učitele.

Příkaz \verb|\NewConfigure| deklaruje konfigurační háčky, které se vkládají do
redefinovaných příkazů. První argument je název konfigurace, druhý je počet
háčků, které budou vytvořeny. Pokud chceme vkládat nějaký kód před příkaz a po
jeho skončení, stačí nám dva háčky. Háčky jsou příkazy ve formátu
\verb|\[a-i]:název háčku|, v našem případě jsou vytvořeny dva,
\verb|\a:teacherinfo| a \verb|\b:teacherinfo|. Pomocí příkazu \verb|\let| si
můžeme uložit původní definici příkazu, který chceme upravit a poté použít
například příkaz \verb|\renewcommand| pro úpravu definice. 

Nakonec konfigurujeme naše háčky promocí \verb|\Configure{teacherinfo}|,
poznámky pro učitele označíme elementem \texttt{span} s atributem
\texttt{epub:type}, který Epub~3 používá pro sémantické značkování textu. Tento
konkrétní typ je součástí profilu \textit{Epub for Education}.

Naše ukázková publikace tedy bude mít dvě verze, jednu pro studenty a druhou
pro učitele. Pro každou z nich vytvoříme řídící dokument, který bude obsahovat
pouze hlavičku dokumentu, samotný text se bude vkládat a bude pro obě verze
stejný. 

Řídící soubor pro učitele \texttt{teacher.tex}

\begin{verbatim}
\documentclass{article}
...
\usepackage[teacher]{edupub} 
\begin{document}
\include{text}
\end{document}
\end{verbatim}

Verze pro studenta bude totožná, pouze balíček \texttt{edupub} vložíme bez
volby \texttt{teacher}. Samotný text dokumentu může mít následující obsah:


\begin{verbatim}
\newcommand\odpoved[1]{\teacherinfo{\\ #1}}
\begin{enumerate}
\item Jak se jmenuje největší had vyskytující se v České republice?
\odpoved{Užovka stromová}
\end{enumerate}
\end{verbatim}

\section{Závěr}
Toto byl pouze jednoduchý příklad, na kterém jsem chtěl ukázat základní
konfiguraci TeX4ht a pokročilé možnosti, které nabízí formát Epub~3. 
Pokročilá konfigurace TeX4ht je složité téma a současný vývojový tým tohoto
projektu se stále nachází spíše ve stadiu digitální archeologie, protože pro
mnoho vlastností neexistuje dokumentace a princip jejich fungování se zjišťuje
zkoumáním zdrojových kódů. Pokud byste se chtěli zapojit do projektu a
například přidat podporu pro nový balíček, nebo nahlásit chybu na kterou jste
narazili při zpracování svých dokumentů, můžete se zaregistrovat na mailovou
konferenci nebo na issue tracker na stránkách
\url{https://www.tug.org/tex4ht/}. Mnoho konkretních příkladů konfigurace lze
nalézt také na serveru TeX.sx, pod heslem \texttt{tex4ht}:
\url{http://tex.stackexchange.com/questions/tagged/tex4ht} 



\end{document}
