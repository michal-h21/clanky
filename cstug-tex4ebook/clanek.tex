# osnova

## úvod do ebooků


Elektronické knihy (e-booky) se rozvíjejí především od 70.~let 20.~století, kdy
byl spuštěn projekt Guttenberg, jehož cílem je digitalizace knih a textů, na
které se nevztahují autorská práva. Prudký rozvoj nastal od 90. let, kdy se
objevily osobní digitální asistenty (PDA) a masověji se rozšířily osobní
počítače. Zpočátku je tvořili především nadšenci, komerční vydavatelství pouze
pomalu ztrácela nedůvěru. V roce 2004 se objevila první čtečka s displejem s
elektronickým inkoustem (e-ink), o tři roky později internetový obchod Amazon
představil svojí čtečku Kindle a nastal velký komerční rozvoj.

Hlavními vlastnostmi e-booků jsou flexibilita zobrazení, neboť je třeba, aby
šly zobrazit na řadě zařízení s různou velikostí a vlastnostmi displeje,
možnost upravit vlastnosti zobrazení, jako je velikost nebo barva písma podle
potřeb konkrétního čtenáře. Některé čtečky také umožňují předčítání textu
technologií text to speech.

V průběhu vývoje se objevilo množství formátů e-booků, v dnešní době jsou
nejrozšířenějšími Epub, z něho vycházející Epub 3 a Mobi, používaný Amazonem.
% Formát PDF se často nepovažuje za e-bookový formát, protože 
Tyto formáty v podstatě vychází z formátů HTML a CSS, běžně používaných na WWW
stránkách, a pomocných metadat. 





## kompilace
## konfigurace
## make4ht
